\chapter{Test Scripts}\label{magicscripts}
\begin{mdframed}
    \begin{python}
from adb_android import adb_android
from time import sleep
from audioshift import run_test

offset = 0
def do_test():
    sleep(5)
    #Actually test here
    for i in range(5):
        res = run_test(5)
        print("Test {} {}:{}ms".format(i, offset, res[0][1]), flush=True)

#Initial control
do_test()

#Test positive offsets
for i in range(5):
    sleep(1)
    adb_android.shell("input tap " + str(0x1af) + " " + str(0x6a9))
    sleep(10)
    offset += 1
    do_test()
    \end{python}
    \captionof{listing}{Test script for AmpMe for the playback control command next.}
    \label{ampme_next}
\end{mdframed}
\clearpage

\begin{mdframed}
    \begin{python}
from adb_android import adb_android
from time import sleep
from audioshift import run_test

offset = 0
def do_test():
    sleep(5)
    #Actually test here
    for i in range(5):
        res = run_test(5)
        print("Test {} {}:{}ms".format(i, offset, res[0][1]), flush=True)

#Initial control
do_test()

#Test positive offsets
for i in range(7):
    adb_android.shell("input tap " + str(0x22d) + " " + str(0x570))
    sleep(1)
    adb_android.shell("input keyevent 22") #Get focus
    adb_android.shell("input keyevent 22") #Get focus
    adb_android.shell("input keyevent 22")
    sleep(1)
    adb_android.shell("input tap " + str(0xae) + " " + str(0x49a))
    offset += 1
    do_test()

#Reset offset
adb_android.shell("input tap " + str(0x22d) + " " + str(0x570))
sleep(1)
adb_android.shell("input keyevent 22") #Get focus
adb_android.shell("input keyevent 22") #Get focus
for i in range(7):
    adb_android.shell("input keyevent 21")
    offset -= 1
sleep(1)
adb_android.shell("input tap " + str(0xae) + " " + str(0x49a))

#Halfway control
do_test()

#Test negative offsets
for i in range(7):
    adb_android.shell("input tap " + str(0x22d) + " " + str(0x570))
    sleep(1)
    adb_android.shell("input keyevent 22") #Get focus
    adb_android.shell("input keyevent 22") #Get focus
    adb_android.shell("input keyevent 21")
    sleep(1)
    adb_android.shell("input tap " + str(0xae) + " " + str(0x49a))
    offset -= 1
    do_test()

#Reset offset
adb_android.shell("input tap " + str(0x22d) + " " + str(0x570))
sleep(1)
adb_android.shell("input keyevent 22") #Get focus
adb_android.shell("input keyevent 22") #Get focus
for i in range(7):
    adb_android.shell("input keyevent 22")
    offset += 1
sleep(1)
adb_android.shell("input tap " + str(0xae) + " " + str(0x49a))

#End control
do_test()
    \end{python}
    \captionof{listing}{Test script for AmpMe manual offset adjustment.}
    \label{ampme_manualoffset}
\end{mdframed}
\clearpage

\begin{mdframed}
    \begin{python}
from scipy.io import wavfile
import numpy as np
from numpy.fft import fft, ifft, fftshift
import begin
from itertools import combinations
from glob import glob
import os.path
from matplotlib import pyplot as plt
import sounddevice as sd


def _cross_correlation_using_fft(x, y):
    """
    Make cross correlation between two signals using fft
    :param x: Sample 1
    :param y: Sample 2
    :returns: Cross correlation of sample 1 and 2
    """
    f1 = fft(x)
    f2 = fft(np.flipud(y))
    cc = np.real(ifft(f1 * f2))
    return fftshift(cc)


def _compute_shift(x, y):
    """
    Compute the time shift of signal y in regards to signal x
    :param x: Sample 1
    :param y: Sample 2
    :returns: < 0 means that y starts 'shift' time steps before x
        > 0 means that y starts 'shift' time steps after x
    """
    assert len(x) == len(y)
    c = _cross_correlation_using_fft(x, y)
    assert len(c) == len(x)
    zero_index = int(len(x) / 2) - 1
    shift = zero_index - np.argmax(np.abs(c))
    return shift

def run_test(time):
    duration = time # seconds
    fs = 48000
    sd.default.samplerate = fs
    audio = sd.rec(duration * fs, channels=2, dtype="float64")
    sd.wait()

    combs = {}
    for pair in combinations(range(audio.shape[1]), 2):
        if not pair[0] in combs:
            combs[pair[0]] = {}
        shift = _compute_shift(audio[:, pair[1]], audio[:, pair[0]])
        combs[pair[0]][pair[1]] = shift / (fs / 1000)
    return combs

@begin.subcommand
def offline():
    print("Offline analysis")
    duration = 5 # seconds
    fs = 48000
    sd.default.samplerate = fs
    audio = sd.rec(duration * fs, channels=2, dtype="float64")
    sd.wait()

    print("  [{} channels, {:.2f} sec. {} Hz]".format(audio.shape[1], audio.shape[0] / fs, fs))
    for pair in combinations(range(audio.shape[1]), 2):
        print("  Ch. {} -> {}:".format(*pair), end='')
        shift = _compute_shift(audio[:, pair[1]], audio[:, pair[0]])
        print(" {:>5} sample(s) [{:>7,.3f} ms]".format(shift, shift / (fs / 1000)))

    sd.play(audio)
    sd.wait()

def callback(indata, outdata, frames, time, status):
    global buffc, buff
    if status:
        print(status, flush=True)
    outdata[:] = indata
    fs = 48000

    audio = indata
    print("  [{} channels, {:.2f} sec. {} Hz]".format(audio.shape[1], audio.shape[0] / fs, fs))
    for pair in combinations(range(audio.shape[1]), 2):
        print("  Ch. {} -> {}:".format(*pair), end='')
        shift = _compute_shift(audio[:, pair[1]], audio[:, pair[0]])
        print(" {:>5} sample(s) [{:>7,.3f} ms]".format(shift, shift / (fs / 1000)))

@begin.subcommand
def online():
    print("Online analysis")
    duration = 10 # seconds
    fs = 48000
    sd.default.samplerate = fs
    with sd.Stream(channels=2, blocksize=1*fs, dtype="float32", callback=callback):
        sd.sleep(duration*10000000)

@begin.subcommand
def file(in_file: 'Wavfile(s) to test. Glob syntax can be used'):
    for file in glob(in_file):
        print("\nFile:", os.path.basename(file), end='')
        rate = 0
        audio = None
        try:
            rate, audio = wavfile.read(file)
        except:
            print("  Not wav file!")
            continue
        print("  [{} channels, {:.2f} sec. {} Hz]".format(audio.shape[1], audio.shape[0] / rate, rate))
        for pair in combinations(range(audio.shape[1]), 2):
            print("  Ch. {} -> {}:".format(*pair), end='')
            shift = _compute_shift(audio[:, pair[1]], audio[:, pair[0]])
            print(" {:>5} sample(s) [{:>7,.3f} ms]".format(shift, shift / (rate / 1000)))

@begin.start(auto_convert=True)
def main():
    print("A")
    \end{python}
    \captionof{listing}{A Python script for calculating cross correlation using \ac{FFT}.}
    \label{mainmagic}
\end{mdframed}
\clearpage

\begin{mdframed}
    \begin{python}
from adb_android import adb_android
from time import sleep
from audioshift import run_test

time = 0
def do_test():
    sleep(5)
    #Actually test here
    for i in range(5):
        res = run_test(5)
        print("Test {} {}:{}ms".format(i, time, res[0][1]), flush=True)

#Initial control
for i in range(5):
    #Next
    adb_android.shell("input tap " + str(0x3fb) + " " + str(0x8ea))
    #Play/Pause
    adb_android.shell("input tap " + str(0x2c2) + " " + str(0x8f4))
    sleep(5)
    #Prev
    adb_android.shell("input tap " + str(0x1aa) + " " + str(0x8f9))
    #Prev
    adb_android.shell("input tap " + str(0x1aa) + " " + str(0x8f9))
    #Play/Pause
    adb_android.shell("input tap " + str(0x2c2) + " " + str(0x8f4))
    sleep(10)
    time = i
    do_test()
    \end{python}
    \captionof{listing}{Test script for SoundSeeder for the playback control command next.}
    \label{soundseeder_next}
\end{mdframed}
\clearpage

\begin{mdframed}
    \begin{python}
from adb_android import adb_android
from time import sleep
from audioshift import run_test

offset = 0
def do_test():
    sleep(5)
    #Actually test here
    for i in range(5):
        res = run_test(5)
        print("Test {} {}ms:{}ms".format(i, offset, res[0][1]))

#Initial control
do_test()

#Test positive offsets
for i in range(10):
    adb_android.shell("input tap " + str(0x3ec) + " " + str(0xa7))
    sleep(1)
    adb_android.shell("input keyevent 22") #Get focus
    adb_android.shell("input keyevent 22")
    adb_android.shell("input tap " + str(0x390) + " " + str(0x49c))
    offset += 40
    do_test()

#Reset offset
adb_android.shell("input tap " + str(0x3ec) + " " + str(0xa7))
sleep(1)
adb_android.shell("input keyevent 21") #Get focus
for i in range(10):
    adb_android.shell("input keyevent 21")
    offset -= 40
adb_android.shell("input tap " + str(0x390) + " " + str(0x49c))

#Halfway control
do_test()

#Test negative offsets
for i in range(10):
    adb_android.shell("input tap " + str(0x3ec) + " " + str(0xa7))
    sleep(1)
    adb_android.shell("input keyevent 21") #Get focus
    adb_android.shell("input keyevent 21")
    adb_android.shell("input tap " + str(0x390) + " " + str(0x49c))
    offset -= 40
    do_test()

#Reset offset
adb_android.shell("input tap " + str(0x3ec) + " " + str(0xa7))
sleep(1)
adb_android.shell("input keyevent 22") #Get focus
for i in range(10):
    adb_android.shell("input keyevent 22")
    offset += 40
adb_android.shell("input tap " + str(0x390) + " " + str(0x49c))

#End control
do_test()
    \end{python}
    \captionof{listing}{Test script for SoundSeeder manual offset adjustment.}
    \label{soundseeder_manualoffset}
\end{mdframed}
\clearpage

\begin{mdframed}
    \begin{python}
from adb_android import adb_android
from time import sleep
from audioshift import run_test

time = 0
def do_test():
    sleep(5)
    #Actually test here
    for i in range(5):
        res = run_test(5)
        print("Test {} {}:{}ms".format(i, time, res[0][1]), flush=True)

#Initial control
for i in range(5):
    adb_android.shell("input tap " + str(0x37d) + " " + str(0x79))
    sleep(1)
    time = i
    do_test()
    \end{python}
    \captionof{listing}{Test script for SoundSeeder resync feature.}
    \label{soundseeder_resync}
\end{mdframed}
\clearpage
