%!TEX root = ../main.tex

\usepackage[english]{babel}
\usepackage[T1]{fontenc}
%TODO: There's something wrong here. Roboto causes fvextra to fail
%Maybe find another font?
%\usepackage[thin, type1]{roboto}
\usepackage{inconsolata}
\usepackage{multicol}
\renewcommand{\multicolsep}{\itemsep}
\usepackage{microtype}
\usepackage[inline]{enumitem}
\usepackage{csquotes}
\usepackage{graphicx}
\usepackage[english]{varioref}
\usepackage{hyperref}
\hypersetup{%
    plainpages=false,%
    bookmarksnumbered=true,%
    colorlinks,%
    citecolor=aaublue,%
    filecolor=aaublue,%
    linkcolor=aaublue,% you should probably change this to black before printing
    urlcolor=aaublue,%
    pdfstartview=FitH%
}
\usepackage[english, capitalize]{cleveref}
\usepackage{import}
\usepackage{booktabs}
\crefname{appchap}{appendix}{appendices}
\Crefname{appchap}{Appendix}{Appendices}
\newcommand{\myref}[1]{\vref{#1}}
\newcommand{\Myref}[1]{\Vref{#1}}

%Titles and captions
\usepackage{titlesec, blindtext, color}
\titleformat{\part}[display]{%
\fontsize{38pt}{8pt}\filcenter\sffamily\selectfont\thispagestyle{empty}}
{\textcolor{part}{Part \thepart \vspace{6pt}\linebreak {\rule{120pt}{1pt}} \linebreak}}
{0pt}
{}
\titleformat{\chapter}[hang]{\Huge\sffamily\mdseries\vspace{-40pt}}{\textcolor{numbers}{\thechapter\hsp{|}\hsp}}{0pt}{\Huge\sffamily\mdseries}[\vspace{-20pt}]

\titleformat{\section}
{\normalfont\Large\sffamily}
{\color{numbers}\thesection}{1em}{}

\titleformat{\subsection}
{\normalfont\large\sffamily}
{\color{numbers}\thesubsection}{1em}{}

\titleformat{\subsubsection}
{\normalfont\normalsize\sffamily}
{\color{numbers}\thesubsubsection}{1em}{}

\usepackage[
    left=28mm,% left margin on an odd page
    right=41mm,% right margin on an odd page
]{geometry}

\usepackage{caption}
\usepackage{subcaption}
\captionsetup{%
    font=footnotesize,% set font size to footnotesize
    labelfont=bf % bold label (e.g., Figure 3.2) font
}
\captionsetup[sub]{%
    font=footnotesize,% set font size to footnotesize
    labelfont=bf % bold label (e.g., Figure 3.2) font
}

%Line numbers
\usepackage[modulo]{lineno}

\usepackage{amsmath}
\usepackage{amssymb}
\usepackage{amsthm}

%Tikz related
\usepackage{tikz}
\usepackage{tkz-graph}
\usepackage{tikz-qtree}
\usepackage{tikzscale}
\usetikzlibrary{%
    shapes,
    shapes.symbols,
    arrows,
    positioning,
    backgrounds,
    matrix,
    patterns,
    calc,
    fit,
    math,
    shapes.multipart,
    automata,
    shadows,
    decorations.pathreplacing
}
\usepackage{pgfplots}
\pgfplotsset{compat=1.12}
\usepackage{packages/tikz-uml}
\usepackage{pgf-umlsd}
%Easy Diagrams
%\usepackage{smartdiagram}
%\smartdiagramset{%
%    set color list={smartdiagram2,smartdiagram1,smartdiagram3,smartdiagram4,smartdiagram5,smartdiagram6},
%    sequence item border color=none,
%    sequence item font size=\small\sffamily,
%}

%Curfile stuff
\usepackage{currfile}
\makeatletter
\def\relativepath{\import@path}
\makeatother

%Headers and footers
\usepackage{fancyhdr}
\pagestyle{fancy}
\fancyhf{}
\renewcommand{\headrulewidth}{0pt} %remove the horizontal line in the header
\fancyhead[RE]{\color{aaublue}\small\nouppercase\leftmark}
\fancyhead[LO]{\color{aaublue}\small\nouppercase\rightmark}
\fancyhead[LE,RO]{\thepage} %page number on all pages
\raggedbottom
\usepackage[para]{footmisc}

%Colors
\usepackage{xcolor}
\definecolor{aaublue}{gray}{0}
\definecolor{bluekeywords}{gray}{0}
\definecolor{greencomments}{gray}{0.5}
\definecolor{redstrings}{gray}{0.3}
\definecolor{codebg}{HTML}{EFEFEF}
\definecolor{codefg}{HTML}{000000}
\definecolor{part}{HTML}{34495E}
\definecolor{numbers}{HTML}{34495E}
\definecolor{smartdiagram1}{HTML}{1ABC9C}
\definecolor{smartdiagram2}{HTML}{2ECC71}
\definecolor{smartdiagram3}{HTML}{3498db}
\definecolor{smartdiagram4}{HTML}{9b59b6}
\definecolor{smartdiagram5}{HTML}{E74C3C}
\definecolor{smartdiagram6}{HTML}{F1C40F}
\definecolor{smartdiagram7}{HTML}{E67E22}
\definecolor{diagramDark}{HTML}{19B5FE}
\definecolor{diagramLight}{HTML}{6BB9F0}

\definecolor{tableGoodLight}{HTML}{87D37C}
\definecolor{tableGoodDark}{HTML}{26A65B}
\definecolor{tableBadLight}{HTML}{EC644B}
\definecolor{tableBadDark}{HTML}{EF4836}

\definecolor{GoogleGreen}{HTML}{4CAF50}
\definecolor{GoogleRed}{HTML}{F44336}
\definecolor{GooglePurple}{HTML}{9C27B0}
\definecolor{GoogleDeepPurple}{HTML}{673AB7}
\definecolor{GoogleIndigo}{HTML}{3F51B5}
\definecolor{GoogleBlue}{HTML}{2196F3}
\definecolor{GoogleLightBlue}{HTML}{03A9F4}
\definecolor{GoogleCyan}{HTML}{00BCD4}
\definecolor{GoogleTeal}{HTML}{009688}
\definecolor{GoogleLightGreen}{HTML}{8BC34A}
\definecolor{GoogleLime}{HTML}{CDDC39}
\definecolor{GoogleYellow}{HTML}{FFEB3B}
\definecolor{GoogleAmber}{HTML}{FFC107}
\definecolor{GoogleOrange}{HTML}{FF9800}
\definecolor{GoogleDeepOrange}{HTML}{FF5722}
\definecolor{GoogleBrown}{HTML}{795548}
\definecolor{GoogleGrey}{HTML}{9E9E9E}
\definecolor{GoogleBlueGrey}{HTML}{607D8B}

\colorlet{Highlight}{GoogleBlue}

%Listing & Minted
\usepackage{shellesc}
\usepackage{listings}
\lstset{%
    backgroundcolor=\color{white},   % choose the background color; you must add \usepackage{color} or \usepackage{xcolor}
    basicstyle=\footnotesize\ttfamily,        % the size of the fonts that are used for the code
    breakatwhitespace=true,         % sets if automatic breaks should only happen at whitespace
    breaklines=true,                 % sets automatic line breaking
    captionpos=b,                    % sets the caption-position to bottom
    commentstyle=\color{greencomments},    % comment style
    deletekeywords={},            % if you want to delete keywords from the given language
    escapeinside={\%*}{*)},          % if you want to add LaTeX within your code
    extendedchars=false,              % lets you use non-ASCII characters; for 8-bits encodings only, does not work with UTF-8
    frame=single,                    % adds a frame around the code
    keepspaces=true,                 % keeps spaces in text, useful for keeping indentation of code (possibly needs columns=flexible)
    keywordstyle=\bfseries\color{bluekeywords},       % keyword style
    morekeywords={},            % if you want to add more keywords to the set
    numbers=left,                    % where to put the line-numbers; possible values are (none, left, right)
    numbersep=5pt,                   % how far the line-numbers are from the code
    numberstyle=\tiny\color{gray}, % the style that is used for the line-numbers
    rulecolor=\color{black},         % if not set, the frame-color may be changed on line-breaks within not-black text (e.g. comments (green here))
    showspaces=false,                % show spaces everywhere adding particular underscores; it overrides 'showstringspaces'
    showstringspaces=false,          % underline spaces within strings only
    showtabs=false,                  % show tabs within strings adding particular underscores
    stepnumber=1,                    % the step between two line-numbers. If it's 1, each line will be numbered
    stringstyle=\bfseries\color{redstrings},     % string literal style
    tabsize=2,                       % sets default tabsize to 2 spaces
    columns=fullflexible,
    postbreak=\raisebox{0ex}[0ex][0ex]{\ensuremath{\color{red}\hookrightarrow\space}}
}

\usepackage{minted}
\newminted[java2]{java}{frame=leftline, framesep=2mm, linenos, fontsize=\footnotesize, baselinestretch=1.1, autogobble}
\newminted[xmlblock]{xml}{frame=leftline, framesep=2mm, linenos, fontsize=\footnotesize, baselinestretch=1.1, autogobble}
\newminted[bashblock]{shell}{frame=leftline, framesep=2mm, linenos, fontsize=\footnotesize, baselinestretch=1.1, autogobble, breaklines, breakanywhere}
\newmintedfile{java}{frame=leftline, framesep=2mm, linenos, fontsize=\footnotesize, baselinestretch=1.1}
\usemintedstyle{tango}
\usepackage{tcolorbox}

%Acronyms and Fixmes
\usepackage{acronym}
\usepackage{fixme}
\fxsetup{%
	layout={footnote, index},
	innerlayout={inline, index},
	theme=colorsig,
	mode=multiuser,
}

\FXRegisterAuthor{jj}{jje}{Jesper}
\FXRegisterAuthor{tn}{tnn}{Thomas}

%External files
\usepackage{xspace}
\fxsetup{%
    status=final,
}

% chktex-file 01 chktex-file 09 chktex-file 10 chktex-file 17

\hyphenation{Amp-Me}
\hyphenation{You-Tube}

\newcommand{\giraf}[1]{\texttt{#1}\xspace}
\newcommand{\android}[1]{\texttt{#1}\xspace}

\newcommand{\method}[1]{%
    \textbf{#1}\xspace
}

\newcommand{\name}[1]{%
    \textit{#1}\xspace
}

\newcommand{\rendpoint}[1]{\texttt{#1}\xspace}

\newcommand{\hsp}{\hspace{20pt}}

\makeatletter
\newtcbox{\code}{
    on line,
    fontupper=\small\ttfamily,
    boxrule=0.5pt,
    arc=2pt,
    coltext=codefg,
    colback=codebg,
    colframe=codebg,
    boxsep=0pt,
    shrink tight,
    extrude by=2pt
}
\makeatother

\def\checkmark{\tikz\fill[scale=0.4, color=smartdiagram2](0,.35) -- (.25,0) -- (1,.7) -- (.25,.15) -- cycle;}

%Make \Chaptername return print the name of the chapter
\let\Chaptermark\chaptermark
\def\chaptermark#1{\def\Chaptername{#1}\Chaptermark{#1}}
\let\Sectionmark\sectionmark
\def\sectionmark#1{\def\Sectionname{#1}\Sectionmark{#1}}
\let\Subsectionmark\subsectionmark
\def\subsectionmark#1{\def\Subsectionname{#1}\Subsectionmark{#1}}
\let\Subsubsectionmark\subsubsectionmark
\def\subsubsectionmark#1{\def\Subsubsectionname{#1}\Subsubsectionmark{#1}}

\newlist{enumberate}{enumerate}{2}
\setlist[enumberate,1]{label=\itshape \arabic*\upshape)}
\setlist[enumberate,2]{label=\itshape \arabic{enumberatei}.\arabic*\upshape)}
\newlist{eletterate}{enumerate}{2}
\setlist[eletterate,1]{label=\itshape \alph*\upshape)}
\setlist[eletterate,2]{label=\itshape \aplh{eletteratei}.\alph*\upshape)}

\newlist{enumberate*}{enumerate*}{1}
\setlist[enumberate*]{label=\itshape \arabic*\upshape)}
\newlist{eletterate*}{enumerate*}{1}
\setlist[eletterate*]{label=\itshape \alph*\upshape)}
% 1st param: font style of label, 2nd param: text of longest label
\newenvironment{dankscription}[2]{\begin{description}[labelindent=\parindent, labelwidth=\widthof{#1#2}, align=parright, font=\normalfont#1]}{\end{description}}

\newcommandx{\cnameref}[1]{%
\cref{#1} \emph{\nameref{#1}}}

\newcommand{\tblgrpsep}{\noalign{\vspace{.75em}}}

%@Temporary: Style to be determined
\newenvironment{problemstatement}
{%
	\begin{mdframed}[
		outermargin=-11pt,
		linewidth=1pt,
		linecolor=Highlight,
		leftline=true,
	]
}
{%
	\end{mdframed}
}

\makeatletter
\def\relativepath{\import@path}
\makeatother


\usepackage{outlines}

\usepackage{mdframed}
\mdfsetup{%
    topline=false,
    rightline=false,
    leftline=false,
    bottomline=false
}

%Tables with fixed widths
\usepackage{tabu}

%Cells spanning multiple rows
\usepackage{multirow}

\usepackage{longtable}

\usepackage{tabularx}

%Used for problem statement env
\usepackage{changepage}

\sloppy
