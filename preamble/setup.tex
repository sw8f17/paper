%!TEX root = ../main.tex
\usepackage[english]{babel}
\usepackage[T1]{fontenc}
\usepackage{inconsolata}

%Give me those text sizes
\usepackage{lmodern}

\usepackage{multicol}
\renewcommand{\multicolsep}{\itemsep}

%Space between line commands
\usepackage{setspace}

%Control enumerate and friends
\usepackage[inline]{enumitem}

%Line numbers
\usepackage[modulo]{lineno} %Keep before fvextra
\usepackage{fvextra} %Keep this before csquote
%Neat context sensitive quotes
\usepackage{csquotes}
%Cool options for the includegraphics command
\usepackage{graphicx}
\usepackage{lastpage}
\usepackage{rotating}
%Commands with multiple optional args
\usepackage{xargs}

%Lets get the ref working
\usepackage[english]{varioref}
\usepackage{hyperref}
  \hypersetup{
    bookmarksnumbered,
    bookmarksdepth=1,
    % pdfauthor={\groupname},
    % pdftitle={\projecttitle},
    pdfsubject={Student Report, Aalborg University},
    pdfpagelayout=TwoPageRight,
    pdfdisplaydoctitle,
    breaklinks,
    linktoc=all,
    pdfprintscaling=None,
    plainpages=false,
    colorlinks,
    linkcolor=aaublue,
    citecolor=aaublue,
    urlcolor=aaublue,
    filecolor=aaublue
  }
\usepackage[english, capitalize]{cleveref}

%Use relative imports
\usepackage{import}

%Fancy table layout help
\usepackage{booktabs}

%Set some crefs
\crefname{appchap}{appendix}{appendices}
\Crefname{appchap}{Appendix}{Appendices}

%Titles and captions
\usepackage{titlesec, blindtext, color}
\titleformat{\part}[display]%
    {\fontsize{38pt}{8pt}\filcenter\sffamily\selectfont\thispagestyle{empty}}
    {\textcolor{part}{Part \thepart \vspace{6pt}\linebreak {\rule{120pt}{1pt}} \linebreak}}
    {0pt}
    {}

\titleformat{\chapter}[hang]%
    {\Huge\sffamily\mdseries\vspace{-20pt}}
    {\textcolor{numbers}{\thechapter \hsp{\rule[-5pt]{1.5pt}{29pt}}\hsp}}
    {0pt}
    {\Huge\sffamily\mdseries}
    [\vspace{-20pt}]

\titleformat{\section}
{\normalfont\Large\sffamily}
{\color{numbers}\thesection}{1em}{}

\titleformat{\subsection}
{\normalfont\large\sffamily}
{\color{numbers}\thesubsection}{1em}{}

\titleformat{\subsubsection}
{\normalfont\normalsize\sffamily}
{\color{numbers}\thesubsubsection}{1em}{}

%Set some margins
\usepackage[
    left=28mm,% left margin on an odd page
    right=41mm,% right margin on an odd page
%    showframe,
]{geometry}

%Dates
\usepackage{datetime}

%Biblography
\usepackage[
    style=numeric,
    maxnames=1,
    backend=biber,
    defernumbers=true,
    hyperref=true,
    alldates=long,
]{biblatex}
\bibliography{bibtex.bib}

%Customize captions
\usepackage{caption}
\usepackage{subcaption} %Used together with caption

\captionsetup{%
    font=footnotesize,% set font size to footnotesize
    labelfont=bf % bold label (e.g., Figure 3.2) font
}
\captionsetup[sub]{%
    font=footnotesize,% set font size to footnotesize
    labelfont=bf % bold label (e.g., Figure 3.2) font
}

%Apparently all "serious math typesetters" need this for math
\usepackage{amsmath}
\usepackage{amssymb}
\usepackage{amsthm}

%Tikz related
\usepackage{tikz}
\usepackage{tikz-qtree}
\usepackage{tikzscale}
%TODO: Could we clean this up?
\usetikzlibrary{%
    shapes,
    shapes.symbols,
    arrows,
    positioning,
    backgrounds,
    matrix,
    patterns,
    calc,
    fit,
    math,
    shapes.multipart,
    automata,
    shadows,
    decorations.pathreplacing,
    external
}
\tikzexternalize[prefix=figures/]
%\tikzexternaldisable{}
\tikzexternalenable{}
%Pgf plots is nice for making fancy plots
\usepackage{pgfplots}
\pgfplotsset{compat=1.12}
\usepgfplotslibrary{fillbetween}
\tikzset{every picture/.style={/utils/exec={\sffamily}}}

%Two uml packages. I love it
\usepackage{packages/tikz-uml}
\usepackage{pgf-umlsd}

%Curfile stuff
%We need this to figure out the path if a command doesn't support import
\usepackage{currfile}
\makeatletter
\def\relativepath{\import@path}
\makeatother

%Headers and footers
\usepackage{fancyhdr}
\pagestyle{fancy}
\fancyhf{}
\renewcommand{\headrulewidth}{0pt} %remove the horizontal line in the header
\fancyhead[RE]{\color{aaublue}\small\nouppercase\leftmark}
\fancyhead[LO]{\color{aaublue}\small\nouppercase\rightmark}
\fancyhead[LE,RO]{\thepage} %page number on all pages
\raggedbottom

%Colors
\usepackage{xcolor}
\definecolor{aaublue}{gray}{0}
\definecolor{bluekeywords}{gray}{0}
\definecolor{greencomments}{gray}{0.5}
\definecolor{redstrings}{gray}{0.3}
\definecolor{codebg}{HTML}{EFEFEF}
\definecolor{codefg}{HTML}{000000}
\definecolor{part}{HTML}{34495E}
\definecolor{numbers}{HTML}{34495E}
\definecolor{smartdiagram1}{HTML}{1ABC9C}
\definecolor{smartdiagram2}{HTML}{2ECC71}
\definecolor{smartdiagram3}{HTML}{3498db}
\definecolor{smartdiagram4}{HTML}{9b59b6}
\definecolor{smartdiagram5}{HTML}{E74C3C}
\definecolor{smartdiagram6}{HTML}{F1C40F}
\definecolor{smartdiagram7}{HTML}{E67E22}
\definecolor{diagramDark}{HTML}{19B5FE}
\definecolor{diagramLight}{HTML}{6BB9F0}

\definecolor{tableGoodLight}{HTML}{87D37C}
\definecolor{tableGoodDark}{HTML}{26A65B}
\definecolor{tableBadLight}{HTML}{EC644B}
\definecolor{tableBadDark}{HTML}{EF4836}

\definecolor{GoogleGreen}{HTML}{4CAF50}
\definecolor{GoogleRed}{HTML}{F44336}
\definecolor{GooglePurple}{HTML}{9C27B0}
\definecolor{GoogleDeepPurple}{HTML}{673AB7}
\definecolor{GoogleIndigo}{HTML}{3F51B5}
\definecolor{GoogleBlue}{HTML}{2196F3}
\definecolor{GoogleLightBlue}{HTML}{03A9F4}
\definecolor{GoogleCyan}{HTML}{00BCD4}
\definecolor{GoogleTeal}{HTML}{009688}
\definecolor{GoogleLightGreen}{HTML}{8BC34A}
\definecolor{GoogleLime}{HTML}{CDDC39}
\definecolor{GoogleYellow}{HTML}{FFEB3B}
\definecolor{GoogleAmber}{HTML}{FFC107}
\definecolor{GoogleOrange}{HTML}{FF9800}
\definecolor{GoogleDeepOrange}{HTML}{FF5722}
\definecolor{GoogleBrown}{HTML}{795548}
\definecolor{GoogleGrey}{HTML}{9E9E9E}
\definecolor{GoogleBlueGrey}{HTML}{607D8B}

\colorlet{Highlight}{GoogleBlue}

%Listing & Minted
\usepackage{shellesc}
\usepackage{listings}

%Minted makes nice looking code listings, but requires listings
\usepackage{minted}
\newminted[java2]{java}{frame=leftline, framesep=2mm, linenos, fontsize=\footnotesize, baselinestretch=1.1, autogobble}
\newminted[python]{python}{frame=leftline, framesep=2mm, linenos, fontsize=\footnotesize, baselinestretch=1.1, autogobble}
\newmintedfile{java}{frame=leftline, framesep=2mm, linenos, fontsize=\footnotesize, baselinestretch=1.1}
\usemintedstyle{tango}
\usepackage{tcolorbox}

\usepackage{mdframed}
\mdfsetup{topline=false, rightline=false, leftline=false, bottomline=false}

%Acronyms and Fixmes
\usepackage{acronym}
%Fixme for configurable notes
\usepackage{fixme}
\fxsetup{%
    layout={footnote, index},
    innerlayout={inline, index},
    theme=colorsig,
    mode=multiuser,
}

\FXRegisterAuthor{jj}{jje}{Jesper}
\FXRegisterAuthor{tn}{tnn}{Thomas}
\FXRegisterAuthor{m}{bigdick}{Marc}
\FXRegisterAuthor{msm}{msme}{Michno}
\FXRegisterAuthor{t}{te}{Troels}
% Eat some spaces
\usepackage{xspace}

%External files
\fxsetup{%
    status=final,
}

% chktex-file 01 chktex-file 09 chktex-file 10 chktex-file 17

\hyphenation{Amp-Me}
\hyphenation{You-Tube}

\newcommand{\giraf}[1]{\texttt{#1}\xspace}
\newcommand{\android}[1]{\texttt{#1}\xspace}

\newcommand{\method}[1]{%
    \textbf{#1}\xspace
}

\newcommand{\name}[1]{%
    \textit{#1}\xspace
}

\newcommand{\rendpoint}[1]{\texttt{#1}\xspace}

\newcommand{\hsp}{\hspace{20pt}}

\makeatletter
\newtcbox{\code}{
    on line,
    fontupper=\small\ttfamily,
    boxrule=0.5pt,
    arc=2pt,
    coltext=codefg,
    colback=codebg,
    colframe=codebg,
    boxsep=0pt,
    shrink tight,
    extrude by=2pt
}
\makeatother

\def\checkmark{\tikz\fill[scale=0.4, color=smartdiagram2](0,.35) -- (.25,0) -- (1,.7) -- (.25,.15) -- cycle;}

%Make \Chaptername return print the name of the chapter
\let\Chaptermark\chaptermark
\def\chaptermark#1{\def\Chaptername{#1}\Chaptermark{#1}}
\let\Sectionmark\sectionmark
\def\sectionmark#1{\def\Sectionname{#1}\Sectionmark{#1}}
\let\Subsectionmark\subsectionmark
\def\subsectionmark#1{\def\Subsectionname{#1}\Subsectionmark{#1}}
\let\Subsubsectionmark\subsubsectionmark
\def\subsubsectionmark#1{\def\Subsubsectionname{#1}\Subsubsectionmark{#1}}

\newlist{enumberate}{enumerate}{2}
\setlist[enumberate,1]{label=\itshape \arabic*\upshape)}
\setlist[enumberate,2]{label=\itshape \arabic{enumberatei}.\arabic*\upshape)}
\newlist{eletterate}{enumerate}{2}
\setlist[eletterate,1]{label=\itshape \alph*\upshape)}
\setlist[eletterate,2]{label=\itshape \aplh{eletteratei}.\alph*\upshape)}

\newlist{enumberate*}{enumerate*}{1}
\setlist[enumberate*]{label=\itshape \arabic*\upshape)}
\newlist{eletterate*}{enumerate*}{1}
\setlist[eletterate*]{label=\itshape \alph*\upshape)}
% 1st param: font style of label, 2nd param: text of longest label
\newenvironment{dankscription}[2]{\begin{description}[labelindent=\parindent, labelwidth=\widthof{#1#2}, align=parright, font=\normalfont#1]}{\end{description}}

\newcommandx{\cnameref}[1]{%
\cref{#1} \emph{\nameref{#1}}}

\newcommand{\tblgrpsep}{\noalign{\vspace{.75em}}}

%@Temporary: Style to be determined
\newenvironment{problemstatement}
{%
	\begin{mdframed}[
		outermargin=-11pt,
		linewidth=1pt,
		linecolor=Highlight,
		leftline=true,
	]
}
{%
	\end{mdframed}
}

\makeatletter
\def\relativepath{\import@path}
\makeatother


%Outlines, apparently nice for nestes lists
\usepackage{outlines}

%Tables with fixed widths
\usepackage{tabu}

%Cells spanning multiple rows
\usepackage{multirow}

%I guess it allows for breaking tables
\usepackage{longtable}

\usepackage{tabularx}
\renewcommand{\arraystretch}{1.8}
%Used for problem statement env
\usepackage{changepage}

%For Chinese characters
\usepackage{luatexja-fontspec}

% For refs to description itmes
\makeatletter
\def\namedlabel#1#2{\begingroup
    #2%
    \def\@currentlabel{#2}%
    \phantomsection\label{#1}\endgroup
}
%DIF UNDERLINE PREAMBLE %DIF PREAMBLE
\RequirePackage[normalem]{ulem} %DIF PREAMBLE
\RequirePackage{color}\definecolor{RED}{rgb}{1,0,0}\definecolor{BLUE}{rgb}{0,0,1} %DIF PREAMBLE
\providecommand{\DIFadd}[1]{{\protect\color{blue}\uwave{#1}}} %DIF PREAMBLE
\providecommand{\DIFdel}[1]{{\protect\color{red}\sout{#1}}}                      %DIF PREAMBLE
%DIF SAFE PREAMBLE %DIF PREAMBLE
\providecommand{\DIFaddbegin}{} %DIF PREAMBLE
\providecommand{\DIFaddend}{} %DIF PREAMBLE
\providecommand{\DIFdelbegin}{} %DIF PREAMBLE
\providecommand{\DIFdelend}{} %DIF PREAMBLE
%DIF FLOATSAFE PREAMBLE %DIF PREAMBLE
\providecommand{\DIFaddFL}[1]{\DIFadd{#1}} %DIF PREAMBLE
\providecommand{\DIFdelFL}[1]{\DIFdel{#1}} %DIF PREAMBLE
\providecommand{\DIFaddbeginFL}{} %DIF PREAMBLE
\providecommand{\DIFaddendFL}{} %DIF PREAMBLE
\providecommand{\DIFdelbeginFL}{} %DIF PREAMBLE
\providecommand{\DIFdelendFL}{} %DIF PREAMBLE
