\chapter{Problem statement}\label{cha:problem_statement}
Our initial problem statement in \cref{sec:initial_problem} refers to the ability to seamlessly synchronize audio playback across multiple devices.
Through \cnameref{cha:sota} we investigated applications that enable this for smart phone devices, while also exploring proprietary solutions with both software and hardware which use this concept of synchronized playback to produce multi-room audio setups.

We concluded that contrary to the proprietary solutions, the current mobile applications on the market are not solving the problem to a satisfactory degree.
This fact, in conjunction with the ease of developing a mobile application compared to a proprietary software and hardware solution, directs our focus on smart phones.

Because of the availability of Android devices in our project group, and no need for special additional hardware, we also choose to develop for Android and not iOS or Windows Phone.

In \cnameref{cha:establishing_use_cases}, we envision different use case scenarios, where some form of manipulated playback of audio on multiple devices could be useful.

\bigskip
All of the above can be summarized as the following problem, which will be examined throughout the rest of this paper.

\begin{problemstatement}
    How can an Android application be used to achieve manipulated playback, wirelessly, across multiple smart phones, such that psychoacoustic effects can be utilized to enhance playback?
\end{problemstatement}

\bigskip\noindent
As in \cnameref{cha:introduction} we introduce some new terms, whose definitions in the context of this paper we define as follows:

\begin{description}
    \item[Manipulated Playback] \hfill \\
        The term \textit{manipulated playback} means that the audio is not simply broadcasted to a myriad of devices and played as soon as the device receives the data.
        Instead we want to control when the playback happens, therefore we define it as manipulated playback and not synchronized playback.
    \item[Psychoacoustic Effects] \hfill \\
        This manipulation of the audio playback, will also enable us to achieve what is called \textit{psychoacoustic effects}, which means that we can utilize the way the human brain perceives sound to our advantage.
        This concept of psychoacoustic effects will be expanded upon in \cnameref{sec:psychoacoustic_effects}.
\end{description}
