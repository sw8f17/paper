\section{Android Sound Stack}
In this section we describe the Android sound stack.
First we will go through the stack and briefly describe the different parts. 
Then we focus on the parts of the stack which is relevant for us.
%Noget indledende tekst forklarende hvorfor vi ser på Android sound stacken
%Noget overblik over sektionen

\subsection{The Sound Stack}
The Android sound stack is all the different components which makes up the sound system on Android devices.
On \cref{} the full sound stack can be seen.

On the top of the stack is the Application Framework.
The frameworks an app, coded in Java, use when dealing with sound is a part of the Application Framework,
for instance the \code{android.media.*} frameworks.

Below the Application Framework is the \ac{JNI}.
The \ac{JNI} makes it possible for Java code to call and be called by the native applications\cite{jni}.

Below the \ac{JNI} is the Native Framework.
The Native Framework is used for implementing apps which runs natively on the system,
for instance an app written in C++. 

Below the Native Framework is the Binder IPC Proxies.
The purpose of the Binder IPC Proxies is to facilitate communication over process boundaries.

Below the Binder ICP Proxies is the Media Server.
The Media Server contains the audio services, which is the code that interacts with the \ac{HAL} layer,
right below the Media Server, in the sound stack.
The sound server implementation on Android is called AudioFlinger, and runs within the Media Server process\cite{audioflinger}.


%Et billede af soundstacken, lille tekst til hver trin i stacken.


\subsection{Application Framework}
\subsection{Native Framework}
\subsection{Media Server}
\subsection{Linux Kernel}
\subsection{Summary}
%Hvad fandt vi ud af.

%https://source.android.com/devices/audio/index.html Main source, dog skal mange af de enkelte punkter findes mere dybdegående information om andetsteds. 
%http://androidsexample.blogspot.dk/2015/01/android-audio-architecture.html Kun som inspiration til en TikZ figur
%https://source.android.com/devices/

