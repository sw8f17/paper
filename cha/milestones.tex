\section{Milestones}\label{sec:milestones}
%Meta, what are milestones and why do we use them.
%What - milestones are small versions of the ideal app, each milstone fulfill a subset of requirements and implement a subset of features, the combined set of milstones encapsulates all requirements and features
%Why - fits with iterative, allows us to always have a working version, we can extend features iteratively, we can divide requirements into realistic and dreams
In order to partition the app into smaller manageable parts we create what we call milestones.
We define a milestone as a functional constituent of the ideal app, each milestone must satisfy a subset of requirements and add or improve features from the former milestones.
Creating these milestones provide us with more tangible small goals; each milestone acts as a constituent of the application.
Thus every time a milestone is completed the application is improved, this provides an iterative structure where we can create milestones ranging from simple and feasible milestones to very complex and time consuming milestones and segment the requirements into these categories.

%List of Milestones
%Milestone Description, argue for the creation of a milestone
    %What requirements are completed by this milestone
\begin{description}
    \item [Basic Audio Player] \hfill \\
        The first milestone is to make a simple audio player able to play local audio files and display the information of the file.
        A simple audio player includes only the most basic control features such as skip, play, pause, and volume control.
        With the success of this milestone the following requirements would be fulfilled: \textit{a, d, h, j}.
    \item [Multi--Device Audio Player] \hfill \\
        With a working basic audio player the application can be extended to multiple devices.
        This includes limiting control from slaves such that they can not skip or change songs.
        The successful implementation of this milestone would fulfill the following requirements: \textit{b, c, e}.
    \item [Basic Synchronization] \hfill \\
        The synchronization is the most central part of the application, we have split this into two separate milestones, the basic synchronization encapsulates reaching the same level of synchronization as the applications in \cnameref{sec:sota_apps} did.
        Reaching this milestone would fulfill the following requirements: \textit{g}.
    \item [Manipulating Playback Offset] \hfill \\ %I dont know in which order we want to go for this or advanced synchronization, they can be swapped without any issues - personally i think this is the correct order.
        Regardless of how well we synchronize the audio playbacks will never be exactly the same.
        In order to adjust for any delays the application should support manipulating playback audio on each device, both automatically and manually.
        This is also required for us to fulfill the problem statement, as such completing this milestone would not only fulfill the following requirements, but also be considered minimum viable product: \textit{f}.
    \item [Advanced Synchronization] \hfill \\
        We consider the advanced synchronization to be anything better than the applications we examined in \cref{sec:sota_apps}.
        This milestone would not fulfill any new requirements, but it would improve the overall quality of our application.
    \item [Psychoacoustic Effects] \hfill \\
        Our large milestone is to reach a level of synchronization and audio playback manipulation where we can be precise enough to induce psychoacoustic effects upon users.
        Psychoacoustic effects covers a wide range of features from the precedence effect, which essentially amplifies volume, to 3D sound manipulation.
        While achieving multiple psychoacoustic effects would create a marketable application, the theory and time to do so, particularly for 3D sound manipulation is not feasible, as such we our goal for this milestone, and ultimately our ideal application goal, would be to successfully utilize the precedence effect.
    \item [Advanced Audio Player] \hfill \\
        This milestone we consider less relevant as our primary focus is on synchronizing devices and manipulating playback, not on the utilities offered as an audio player.
        This milestone concerns adding features you would expect from an audio player such as playlists, queues and support for external sound sources such as Spotify integration.
        The implementation of these features would fulfill the following requirements: \textit{i, k}.
\end{description}



%Milstones
%BasicBitch
    %Play Audio on local device
%Slavery 101
    %Get someone else to play my shit -leaders delegate
%Slavery Advanced
    %Get multiple people to do my shit
%Slavery, expert version
    %Get the slaves to work in a synchronized manner, better than the competetion at least
%Slavery, optimised
    %Make your fellow slavers jealous, have ur slaves be so synchronized it just seems like 1 really well performing slave.

%Manipulating the offset
    %Start manipulating the offset in such a way as to achieve psychoacoust

%3D bitches
    %Manipulate the sound such that the perceived origin can be manipulated