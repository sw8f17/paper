%%OLD% \chapter{Synchronization}
%%OLD% % Why are we writing about this?
%%OLD% A central element in this project is the ability to synchronize. 
%%OLD% This chapter aims to explore and evaluate different methods which could be used to synchronize several Android devices. 
%%OLD% % What requirements does this stem from?
%%OLD% Several of our requirements are directly related to synchronization, these are \tnote{insæt dem her \ldots}.
%%OLD% % What are some of the issues regarding synchronization?
%%OLD% 
%%OLD% Synchronizing multiple devices is a hard problem, as there is a large amount of hardware and software between each of the devices in form of operating systems, driver, wireless chips, and various layers %%OLD% of abstraction, which all introduce either latency, either predictable or not. 
%%OLD% %Furthermore, even if the devices are synchronized, then the delay from a command is issued by our app, to the time the audio playback start  
%%OLD% Furthermore, there is a delay from when the command to play sound in software is issued, to when the actual playback occurs, which depends on the hard- and software of the smartphone. 
%%OLD% That is, that the delay might differ from smartphone to smartphone to such a degree that we need to take it into account. 
%%OLD% 
%%OLD% % What ways of sync do we imagine that exists?
%%OLD% 
%%OLD% We have identified four categories of methods to synchronize the devices:
%%OLD% \begin{itemize}
%%OLD%    \item Using a network connection,
%%OLD%    \item Using the GPS,
%%OLD%    \item Using the audio system of the devices,
%%OLD%    \item Using light, i.e. camera, screen and/or LED.
%%OLD% \end{itemize}
%%OLD% % http://ieeexplore.ieee.org/document/1424575/?arnumber=1424575&tag=1
%%OLD% 
%%OLD% % What way is "best"?
%%OLD% % How will we do it? 
\section{Synchronization}
%What is the importance of synchronization to us
    %hope for the best is not feasable
    %To make psychoacoustic effects reality we need farily accurate manipulation depending on effect
    %Audible sync is not the same as clock synchronization, a slight desync in clocks may actually cause psychoacoustic effects (precedence effect), essentially providing us with an acceptable threshold for desynchronization.
Synchronization is an essential part of the application we are developing;as simply sending playback commands between devices and hoping they are received fast enough to not disrupt the audible synchronization, is not a reliable solution.\mnote{What a sentence}
In order for us to achieve any kind of audible synchronization we need a somewhat accurate clock synchronization across multiple devices.
The accuracy of the synchronization will also determine the degree to which we can manipulate playback to achieve psychoacoustic effects should we reach this milestone.
This section explores and evaluates the methods we have considered using.
%Why is sync of such importance?

An important distinction to make when discussing synchronization for our application is the difference between audible synchronization and clock synchronization.
This section focuses clock synchronization, the better clock synchronization we acquire the more accurate our playback manipulation becomes, and by extension our ability to create audible synchronization.
In that audible synchronization is the goal for the application, this also means that precision in the range of milliseconds is sufficient, furthermore one of our requirements mentioned in \cnameref{cha:requirement_elicitation} is to compete with \ac{SOTA}.

%How can we achieve this synchronization
    %We propose and have looked into several methods
        %GPS
        %Audio
        %Light cues
        %Messages/Protocols(my time, adjust to this)
            %We looked into some academic research on this, however found that it was mostly unimplemented theories, often a bit off the mark from what we wanted to do.
        %NTP(external sync/offset calculations)
\subsection{Methods}
The following is a list of the possible solutions we have considered:
\begin{itemize}
    \item GPS
    \item Audio
    \item Light cues.
    \item Timestamps\mnote{Could expand this here already(NTP, PTP, clock sync to master etc.)}
\end{itemize}

\subsubsection{GPS}
%https://spectracom.com/resources/essential-education/gps-clock-synchronization
GPS satellites contain multiple atomic clocks.
These clocks are monitored and controlled to keep them synchronized with the UTC standard.
This provides us with an accurate clock to synchronize towards.
The synchronization could be done in two ways.
\subsubsection{Audio}
\subsubsection{Light Cues}
\subsubsection{Timestamps}

%What are the primary issues regarding synchronization.
    %msg delay
        %using offsets can help
%Perfect syncrhonization may actually cause problems, audible sync != clock sync

\subsection{Choice/Conclusion}
%What way is "best"
%What seems feasable/which one are we attempting
%could we merge/utilize elements from multiple ideas to acquire a superior form of synchronization
%This would solve "basic" synchronization milestone, how does it affect advanced synchronization milestone, are other solutions better for this? in that case why not start with those?