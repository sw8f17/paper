In this chapter we present different options for internal architectures of our Android app.
By internal architecture we mean how the parts of our Android app will be composed, and how those parts will interact.
We deem it important to choose an architecture, which will allow for extensibility, maintainability and testability;
extensibility, because we want to be able to easily extend the functionality of the app during an agile workflow, this works particularly well with our milestone approach in that we segment basic and advanced synchronization into two parts;
maintainability because we want to have the ability to easily rewrite old code and fix bugs;
and testability because we want to produce a well tested Android app without having to spend all our time testing.

All proposed internal architectures, will be evaluated in regard to how well these requirements are met, and lastly we choose an architecture to use in the implementation of our Android app.

Choosing a fitting internal architecture is as important as the overall system architecture, and especially when developing something as complex as Android apps.
The remainder of this chapter focuses on choosing an architecture which fits the aforementioned criteria.
