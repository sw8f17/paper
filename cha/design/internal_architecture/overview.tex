\subsection{Candidates for Internal Architecture}
In this section we will present three proposed internal architectures.
We define the internal architecture as the architecture of the application device in its own instance, i.e.\ how the separate components that make up the application are structured.
The architectural patterns we will consider in this section are chosen from a selection of patterns in Ian Sommerville's \textit{Software Engineering}\cite[p. 175-184]{sommerville}, where they are presented as architectural patterns commonly used.
We then further picked the three patterns most applicable for our App, which are:
\begin{eletterate*}
    \item Repository architecture;
    \item \ac{MVC}; and
    \item Client-Server architecture.
\end{eletterate*}
These architectural patterns are often used when designing full systems, but can be adapted to the inner workings of components in a system, such as an Android app in our case.

\subsubsection{Repository Architecture}
%all data is managed by single repository
%many components, never interact directly only through repo
%e.g. an IDE
%independent components
%single point of failure
The repository architecture, as described in \textit{Software Engineering}\cite[p. 179-180]{sommerville}, consists of a single centralized \enquote{repository} which manages all data.
This data is then accessible by all components in the system, thereby restricting any component to component interaction.

When implementing the repository architectural pattern, the different components can be developed completely independently, and do not need to know anything about each other.
This results in extremely low coupling; components can be added and removed without breaking the system, and thoroughly tested independently.
An example of applications with the repository architecture, are IDEs; they consist of a single repository containing all project data, and multiple components interacting with the data such as code generators, editors, and analyzers.

A disadvantage of the repository architectural pattern is, that the having a centralized repository is a single point of failure.
If the repository fails or corrupts data, there is no failsafe mechanism.
Additionally, this architecture is used in systems that needs to store a large amount of data for a long time, which does not fit our use case as we barely have any data to store.
For us this would mean that the core building block of the system, the repository, would serve next to no purpose as we hardly have any data to manage.

\subsubsection{\acl{MVC}}
This architectural pattern, also presented in \textit{Software Engineering}\cite[p. 176]{sommerville}, seeks to separate interaction and presentation from the data, thereby the name; \textit{model}, the data, \textit{view}, the presentation, and \textit{controller}, the interaction.
The pattern allows for data to change form without necessarily affecting the presentation or interaction and vice versa.
It is a pattern that yields low coupling, and it can help us ensure that the code regarding \ac{UX} does not interfere with the business logic.
This means that the maintainability and especially the testability, of code adhering to the \ac{MVC} architectural pattern, is significantly higher than code without a clear separation of model view and controller.

A downside to using the MVC pattern is that simple data interactions can become complex and require additional code in order to keep view and model completely separated.
MVC is particularly helpful for apps with loads of interaction, which our app does not have, rather we want ours to simply run in the background which in itself makes MVC match poorly with the application we want to obtain.
Furthermore, there needs to be a clear data model, something our Android app does not have, essentially leaving control as the sole focus should we choose this architecture.

\subsubsection{Client-Server Architecture}
Normally when addressing a client-server architecture, it is in the context of web applications consisting of a client running in the browser which connects to a server running application code.
In our case, where we talk about the internal architecture of an Android app, the server is a service, i.e.~some components which runs separately and can be connected to.
The Client-Server pattern is described in \textit{Software Engineering}\cite[p. 180-181]{sommerville}.
The clients are only used to make the services accessible by the user, as the services does not depend on the clients, which means that services can be developed and tested without being influenced by user interfaces.

Moreover, this architectural pattern allows for a myriad of different clients to use a given service, such that the same functionality can be available in different scenarios.
E.g.~if playback of music is implemented as a service, a local client driven by the user could change songs just as easily as a client controlled remotely.
This would give a seamless experience of control, since both clients are connected to the same service; meanwhile, they do not need to know about each others' existence.

The separation of client and service also allows for more focused testing, since the components can be tested completely independent and the in conjunction.
In our case the client-server architecture lies somewhere between the two previously presented architectures, and gives a middle ground where we do not need a heavy focus on data.
However, because we want to disconnect client functionality from presentation, to improve testability and maintainability, the concept of a client in the client-server architecture, will have to be designed and developed with a more fine grained aspect.

