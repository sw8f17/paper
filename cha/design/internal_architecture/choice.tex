\subsection{Choosing The Internal Architecture}
We will be picking certain concepts from different architectural patterns, to achieve a tailor made internal architecture.

We deem that the internal architecture should have the client--server architecture as foundation.
This means that the internals of our Android app will consist of services and clients;
where the services provides features, and the clients accesses these features and exposes them to different parts of the system or the user.
The advantage of this choice is, that we should be able to extend the functionality by introducing new services or adding to existing ones, and then implement said functionality by developing clients.

This also adheres well to the philosophy of android development, where things like \textit{services} and \textit{activities} are core building blocks, more about these will be explained later in \cnameref{subsec:activities} and \cnameref{subsec:services}.

However, because we are developing an Android app, which inevitably will have some form of \ac{UI}, we want to incorporate the idea of views from the \ac{MVC} pattern.
These views will be anything that is to be presented to the user, such as playlists or playback controls.
This further separation will enable us to decouple client functionality from client presentation, and allow for testing of the clients functionality independently from the graphical presentation.
