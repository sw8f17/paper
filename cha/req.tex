

Technical requirements:

\begin{eletterate}
    \item \textbf{Run on Android 5.0+} \hfill\\
        The app has to be able to run on Android versions 5 and above.

    \item \textbf{At least two devices must be able to connect to the same session wirelessly.} \hfill\\
        Two or more devices have to be able to connect to each other wirelessly.
        They connect through a ``session'', which is a master/slave relation, where only one device is the master, the rest are slaves.
        This session makes it possible to transmit data from the master to the connected slaves.
        The device which creates the session is automatically the master, slaves can find already created sessions and connect to one.

    \item \textbf{The device which start the session, i.e. the master, must be able to control the playback, e.g. pause, resume and change media, on all devices.} \hfill\\
        The master is in control of the created session, including the playback of the media.
        This means that the master device can pause and resume the playing media for the connected slaves.
        Furthermore the master can also change the media being played, skip media or completely stop the playback of media.

    \item \textbf{Each user must be able to change their own volume.} \hfill\\
        A user must be able to control and change the volume his device when it is connected to a session.
        This possibility applies for both the master device and the connected slaves.
        The change must not have an effect on the other devices connected to the session.

    \begin{itemize}
        \item \textbf{Each user can temporarily or permanently mute their own audio.} \hfill\\
            A user must be able to mute the playback when it is connected to a session.
            This possibility applies for both the master device and the connected slaves.
            The change must not have an effect on the other devices connected to the session.
        
        \item \textbf{Manipulate playback on all devices within the same session.} \hfill\\
            The master must be able to manipulate the playback on all devices connected to its session.
            Manipulate means to control the playback offset between the devices.\tnnote{This can maybe be specified further.}
    
        \item \textbf{The app must have the ability to perform an automatic synchronization.} \hfill\\
            When a device is connected to a session, the app must be able to synchronize the playback automatically,
            so the devices play in sync.
    
        \item \textbf{Following an automatic synchronization, the audio offset between any two devices must not exceed $x$ ms.} \hfill\\
            With the ability to synchronize automatically, the audio offset between any two devices must not exceed $x$ ms.
            $x$ is the maximum desynchronization that will not be noticed by a listener.
    
        \item \textbf{The change in offset following a change in media must not exceed $y$ ms.} \hfill\\
            When the media is changed, for instance a change in song, the offset change must not exceed $y$ ms.
            This means that the media change must be consistent to the extend of $y$ ms.

        \item \textbf{The change in offset after pausing, and resuming playback of any media must not exceed $y$ ms.} \hfill\\
            When the media is paused and resumed by the master, the offset change must not exceed $y$ ms.
            This means that the pause and subsequently resuming must be consistent to the extend of $y$ ms.

        \item \textbf{The drift must not exceed $z$ ms over $n$ seconds.} \hfill\\
            When the media have been playing for $n$ seconds, the drift must not exceed $z$ ms.

    \end{itemize}
\end{eletterate}
UI/UX requirements:
\begin{eletterate}[resume]
    \item \textbf{Display song information (artist, title, duration/length)} \hfill\\
        The song information must be displayed 
    \item \textbf{Have support for playlists (with shuffle, repeat button and queue functionality).} \hfill\\
    \item \textbf{Support Spotify as an audio source.} \hfill\\
    \item \textbf{Play mp3 audio files from the master device local storage.} \hfill\\
\end{eletterate}
