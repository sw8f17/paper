\chapter{Requirement Elicitation}\label{cha:requirement_elicitation}
To elicit requirements for the design and decelopment process of a solution, we focus on the loose requirements posed in \cnameref{part:introduction}.
Each requirement will be described and elaborated upon, and assigned to one of two categories which are \textbf{technical requirements} and \textbf{UI/UX requirements}.

Firstly, we describe and present the requirements from the category of \textbf{technical requirements}.

\begin{eletterate}
    \item \textbf{Run on Android 5.0 and newer} \hfill\\
        As mentioned in \cnameref{cha:problem_statement}, Android is chosen since it is the available system in our project group.
        This means that the app have to run on Android version 5.0 and later.
        Version 5.0 is chosen since significant changes happened to the Android sound stack in this version\cite{android_5_sound_stack}.

    \item \textbf{At least two devices\tnnote{device (written in reqiurements) vs smartphone (written in problem statement)} must be able to connect to the same session wirelessly.}\tnnote{Session is first mentioned here.} \hfill\\
        To be able to achieve multiple playback across multiple devices, as mentioned in \cref{cha:problem_statement},
        then two or more devices have to be able to connect to each other wirelessly.
        They connect through a ``session'', which is a master/slave relation, where only one device is the master, the rest are slaves.
        This session makes it possible to transmit data from the master to the connected slaves.
        The device which creates the session is automatically the master, slaves can find already created sessions and connect to one.

    \item \textbf{The device which starts the session, i.e.~the master, must be able to control the playback, e.g.~pause, resume and change media, on all devices.} \hfill\\
        This requirement relates to the use cases described in \cref{sec:category_increase_volume_and_area}\tnnote{This needs to be added explicit in use case.}.
        The master is in control of the created session, including the playback of the media.
        This means that the master device can pause and resume the playing media for the connected slaves.
        Furthermore the master can also change the media being played, skip media or completely stop the playback of media.

    \item \textbf{Each user must be able to change their own volume.} \hfill\\
        If the user does not want to listen to the music as loud as the rest connected to the session, as mentioned in \cref{sec:category_increase_volume_and_area}, 
        then a user must be able to control and change the volume of his device when it is connected to a session.
        This possibility applies for both the master device and the connected slaves.
        The change must not have an effect on the other devices connected to the session.

    \item \textbf{Each user can temporarily or permanently mute their own audio.} \hfill\\
        A user must be able to mute the playback when it is connected to a session, which relates to the use cases found in \cref{sec:category_increase_volume_and_area}.
        This requirement is related to the previous requirement regarding volume control of the devices.
        The possibility of muting applies for both the master device and the connected slaves.
        The change must not have an effect on the other devices connected to the session.
    
    \item \textbf{Manipulate playback on all devices within the same session.} \hfill\\
        As mentioned in the problem statement in \cref{cha:problem_statement}, the app must be able to manipulate the playback of the audio. 
        To specify, the master must be able to manipulate the playback on all devices connected to its session.
    
    \item \textbf{The app must have the ability to perform an automatic synchronization.} \hfill\\
        To be able to make a system as mentioned in \cref{sec:category_increase_volume_and_area}\tnnote{Automatic sync should be mentioned in use case}, 
        the app must be able to automatically synchronize between the devices in the session. 
        This means that when a device is connected to a session, the app must be able to synchronize the playback automatically,
        so the device plays in sync with the rest of the connected devices in the session.

    \begin{itemize}
        \item \textbf{Following an automatic synchronization, the audio offset between any two devices must not exceed $x$ ms.} \hfill\\
            With the ability to synchronize automatically, the audio offset between any two devices must not exceed $x$ ms.
            $x$ is the maximum desynchronization that will not be noticed by a listener.\tnwarning{x, y, z and n must be specified.}

        \item \textbf{The change in offset following a change in media must not exceed $y$ ms.} \hfill\\
            When the media is changed, for instance a change in song, the offset change must not exceed $y$ ms.
            This means that the media change must be consistent to the extent of $y$ ms.

        \item \textbf{The change in offset after pausing, and resuming playback of any media must not exceed $y$ ms.} \hfill\\
            When the media is paused and resumed by the master, the offset change must not exceed $y$ ms.
            This means that the pause and subsequently resuming must be consistent to the extend of $y$ ms.

        \item \textbf{The drift must not exceed $z$ ms over $n$ seconds.} \hfill\\
            When the media have been playing for $n$ seconds, the drift must not exceed $z$ ms.
    \end{itemize}
\end{eletterate}

The following requirements are of the \textbf{UI/UX requirements} category:
\begin{eletterate}[resume]
    \item \textbf{Display song information} \hfill\\
        When users use the app in according to the use cases mentioned in \cref{sec:category_increase_volume_and_area},
        then the song information, i.e. artist, title, duration/length must be displayed.
        Both the master and slaves must be able to see the media information.

    \item \textbf{Have support for playlists.} \hfill\\ 
        The app must have playlists, so media can be put in a list without having to manually select a new song after a song is done,
        which is in relation to the use cases mentioned in \cref{sec:category_increase_volume_and_area}.
        With having playlists, the app must be able to shuffle a playlist, 
        have a repeat button to repeat a specific media and be able to queue media selected by both the master and slaves.

    \item \textbf{Play mp3 audio files from the master device local storage.} \hfill\\
        The app must be able to play audio files of the format mp3 from the local storage on the device.
        mp3 is chosen since Android supports the file format and because mp3 is a compressed file format,
        resulting in a smaller file size\cite{android_mp3_support}\cite{mp3_compression}.

    \item \textbf{Support Spotify as an audio source.} \hfill\\
        The app must be able to use audio from Spotify as source,
        which means that a master device must be able to stream Spotify audio to the connected slaves.
        This requirement is related to the use cases mentioned in \cref{sec:category_increase_volume_and_area},
        where users having a social gathering must have access to music which is not local on the device.
\end{eletterate}
