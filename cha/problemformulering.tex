\chapter{Problem statement}
%Opsummering af pointer fra indledning, sota, test og use cases
Our initial problem statement in \cref{sec:initial_problem} refers to the ability to seamlessly synchronize audio playback across multiple devices.
Through our state of the art chapter we investigated applications that enable this for smartphone devices, while also exploring speaker manufacturer companies which use this concept to produce multi-room audio systems. 
We concluded that contrary to the speaker manufactuers, the current mobile applications on the market are not solving the problem to a satisfactory degree, as such we choose to focus on smartphones.

%Mangler stadig at læse noget om usecase for at skrive det følgende fuldstændigt
%Use cases går beyond smartphones maybe? if so afgrænsning skal flyttes
Envisioning use cases where the audio playback of multiple smartphone devices could be manipulated, the use cases in \cref{cha:establishing_use_cases} were created.
These use cases are not exclusively scenarios where the audio playback must be synchronous, but also adhere other factors to reach certain psychoacoustic effects, such as 3D sound, i.e. manipulating where a listener observes the sound to originate from.

This leads us to formulate the following problem statement:

%Current Best:
\begin{problemstatement}
    How can we design, implement, and test an Android application which manipulates the audio playback of multiple Android devices in order to achieve various psychoacoustic effects, such as synchronous playback?
\end{problemstatement}

The key terms in the problem statement and their contextual meaning is as follows:

\begin{description}
    \item [Psychoacoustic effects] \hfill \\
        In order to fulfil the use cases described in \cref{cha:establishing_use_cases}, we have to manipulate sound in a manner that goes beyond synchronizing audio playback.
        For something to not simply sound synchronous but have the volume amplified, psychoacoustic effects must be utilized.
        Similarly in order to manipulate the perception of the sounds origin, i.e. 3D audio manipulation, other psychoacoustic effects must be utilizd.
        Which psychoacoustic effects will be explored depends on the specific use cases we decide to try to solve, as the project progresses more psychoacoustic effects may be explored, as feature development goes on. 
    \item [Synchronous Playback] \hfill \\
        Synchronous playback has been mentioned frequently throughout the introduction part, however for something to be synchronous and seem synchronous, is not the same.
        As such when we talk about synchronous playback we talk about the perception of the audio being synchronous, not that it technically is synchronous.
\end{description}
