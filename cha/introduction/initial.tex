\section{Initial problem}

The problem described in the introduction, and the one that will be used
in this project as a starting point can be generalized to the following:

\begin{problemstatement}
	How can we make an application that seamlessly synchronizes audio
	between multiple \jjnote{disjoint?} devices?
\end{problemstatement}

The key terms of the initial problem statement, and their contextual meaning,
is as follows:
\begin{description}
	\item[Application] \jjnote*{I don't like we, but what else?}{We}
		imagine that the solution to the problem will include a software
		application. Which kind of application will depend on what specific
		device the usecase demands.
	\item[Seamlessly] From the initial idea generation phase we suspect
		that the skew, or delay between devices, of the audio playback will
		play a significant role in the design of the solution. Therefore we
		want to synchronize the audio to a precision that makes it feel
		seamless
	\item[Synchronizes audio] Synchronization of audio is what initial focus of
		the project, but it is possible that the solution could be trivially
		expanded to synchronizing arbitrary data packets. The necessity of
		being audio specific will have to be further investifarted.
	\item[Multiple devices] \jjnote*{Same as above}{We} envision that
		a solution would involve multiple devices to seemingly amplify the
		signal without the distortion of trying to push sounds sounds out of
		small speakers.
\end{description}

In order to solidify our initial problem statement into an actual problem
statement this paper will analyze the needs of our stakeholders, as well as the
relevant existing solutions.\jjnote{I don't like this, but it's all I got}
