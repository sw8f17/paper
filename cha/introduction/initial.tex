\section{Initial Problem Statement}\label{sec:initial_problem}

The problem described above, and the one that will be used
in this project as a starting point, can be summarized as follows:

\begin{problemstatement}
    How can we make an application that seamlessly synchronizes audio
    playback wirelessly between multiple devices?
\end{problemstatement}

\noindent
In order to minimize the ambiguity of the initial problem statement,
we define and explain the terms we use, in the context of this paper as follows:

\begin{description}
    \item[Seamlessly synchronizes]  \hfill \\
        From the initial idea generation phase we suspect
        that the delay between devices, of the audio playback will
        play a significant role in the design of the solution. Therefore we
        want to synchronize the audio to a precision that makes it feel
        seamless to the human perception.
    \item[Wirelessly]  \hfill \\
        Part of the application being seamless is also that external wires etc.\ are
        not required for the devices to function together.
    \item[Mutliple devices]  \hfill \\
        We envision that a solution would involve multiple devices to effectively
        amplify the signal without the distortion of trying to play loud sound with
        small speakers. The devices need not be the same hardware.
\end{description}

To get a further understanding of the problem at hand, and to
construct a final problem statement, we explore use cases for such a system;
then analyze some existing related systems on the market,
and test some of them in depth.
We do this to better understand the problem, and use that knowledge
to make a final problem statement, and a list of requirements for the system.
