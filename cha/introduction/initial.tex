\section{Initial Problem Statement}\label{sec:initial_problem}

The problem described in the introduction, and the one that will be used
in this project as a starting point can be summarized to the following:

\begin{problemstatement}
    How can we make an application that seamlessly synchronizes audio
    playback wirelessly between multiple devices?
\end{problemstatement}\mnote{Det er devices vi vil sync of ikke playback, er det for tidligt at lave den distinction uden info omkring hvorfor devices > playback?}

\noindent
In order to minimize the ambiguity of the initial problem statement,
we define and explain the terms we use, in the context of this paper as follows:\footnote{Memes are real}

\begin{description}
    \item[Seamlessly synchronizes]  \hfill \\
        From the initial idea generation phase we suspect
        that the delay between devices, of the audio playback will
        play a significant role in the design of the solution. Therefore we
        want to synchronize the audio to a precision that makes it feel
        seamless to the human ear.
    \item[Wirelessly]  \hfill \\
        Part of the application being seamless is also that external wires etc. are
        not required for the devices to function together.
    \item[Mutliple devices]  \hfill \\
        We envision that a solution would involve multiple devices to seemingly
        amplify the signal without the distortion of trying to push sound out of
        small speakers. The devices need not be the same exact hardware.
\end{description}

To get a further understanding of the problem at hand, and to
construct a final problem statement, we explore use
cases for such an application.
Then analyze some existing related solutions on the market,
and test some of them in depth.
We do this to understand the problem, and use that knowledge
to make a final problem statments, and a list of requirements for the project.
