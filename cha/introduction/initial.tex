\section{Initial Problem Statement}\label{sec:initial_problem}

The problem described in the introduction, and the one that will be used
in this project as a starting point can be summarized to the following:

\begin{problemstatement}
	How can we make an application that seamlessly synchronizes audio
	playback between multiple devices?
\end{problemstatement}

The key terms of the initial problem statement, and their contextual meaning,
is as follows:
\begin{description}
	\item[Application] \hfill \\
		We imagine that the solution to the problem will include a software
		application. Which kind of application will depend on what specific
		device the use case demands.
	\item[Seamlessly]  \hfill \\
		From the initial idea generation phase we suspect
		that the skew, or delay between devices, of the audio playback will
		play a significant role in the design of the solution. Therefore we
		want to synchronize the audio to a precision that makes it feel
		seamless to the human ear.
	\item[Synchronizes audio playback]  \hfill \\
		Synchronization of audio is the initial
		focus of the project, but it is possible that the solution could be
		trivially expanded to synchronizing arbitrary data packets. The
		necessity of being audio specific will have to be further investigated.
	\item[Multiple devices]  \hfill \\
		We envision that a solution would involve multiple devices to seemingly 
		amplify the signal without the distortion of trying to push sound out of
		small speakers. The devices need not be the same exact hardware.
\end{description}

To get a further understanding of the problem at hand, and to
construct a final problem statement, we first analyze some related
existing solutions, on the market, and test a few of them in depth. 
Moreover, we identify, categorize and explore some use cases which are relevant for the project,
in order to finalize the requirements of the project.
