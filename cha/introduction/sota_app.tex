\section{Android Apps}
Introduction mombo jumbo
State the fact that it's all sync streaming apps.

\subsection{SoundSeeder}
SoundSeeder\footnote{http://soundseeder.com} is an app made by JekApps. 
It is compatible with Android 4.1 and above, for older versions of Android (2.2 -- 4.0),
another app called SoundSeeder Speaker can be used as speaker, but not to play the music.
SoundSeeder also provides a Java application for compatibility with other platforms e.g. Windows, macOS and Linux.
JekApps do not support iOS and Windows Phone\cite{soundseeder_ios}.

Noget omkring design og usability her.

SoundSeeder streams music via Wi-Fi or the build--in Android hotspot\cite{soundseether_faq}, 
which means that all phones have to be on the same network.
In regard to the music source, it can be Google Music, YouTube (via semperVidLinks app), online radio stations and local media.
The supported media format is depended on the used device and Android version.\cite{soundseether_faq}

SoundSeeder is free to install but the free version is only possible to use with two devices connected as speakers for up to 15 minutes at the time and it do also contains banner adds. 
The full app costs 39.90 DKK. 

\iffalse
3.9 stars on Google Play
Functionality / How good it works
11. november 2016 https://play.google.com/store/apps/details?id=com.kattwinkel.android.soundseeder.player
For Android > 4.1 and Java based devices (RPI, Linux, Windows, macOS etc.), no plans for iOS and Windows Phone http://soundseeder.com/support/topic/ios-and-windows-support-soon/
Google Music, YouTube (via semperVidLinks app), online radiostations, 
Steam music via Wi-Fi or portable hotspot, UPnP/DLNA browser, mp3, mp4, m4a, aac, 3gp, ogg, flac (depends on your device / android version)
Free version: two speaksers for up to 15 min as often you want, it contains banner adds. Priced version: 39,90 DKK
Sync settings: ?
\fi

\subsection{AmpMe}
The second app is AmpMe, made by Amp Me Inc., and is another app to synchronize music between devices.
It supports Android 4.1 or newer and iOS 9.0 or newer.

Noget omkring design og usability her.

To be able to stream music between devices, AmpMe require internet, in the form of WiFi or mobile data.
This means that the connected devices do not have to be on the same network for AmpMe to work. 
As for the music source, AmpMe supports Spotify, SoundCloud, YouTube and local media.
AmpMe is free to use in both Google Play and iTunes.


\iffalse
By Amp Me Inc.
4.2 stars on Google Play, current version, 3.5 stars, all versions 4 stars on iTunes
Functionality / How good it works
30. januar 2017 (Android), Feb 01, 2017 (iOS)
Android and iOS
Spotify, SoundCloud, YouTube and your local music library
Require internet, WiFi/Mobile data
Free
Resync button, syncs upon connection
\fi
\subsection{SpeakerBlast}

\subsection{Chorus}
The final app is called Chorus and is made by AVR APPS.
Is supports Android 4.0 or newer, they state to have an iOS app, but the iTunes page returns that the app is not available in our region,
when open in iTunes\footnote{https://itunes.apple.com/app/chorus/id894014439}.



\iffalse
By AVR APPS 
3.3 stars on Google Play
Functionality / How good it works
11. maj 2015
Android 4.0, iOS
Local media?
Mobile hotspot or WiFi
Free
Manual/automatic sync
\fi
\subsection{}
\iffalse
        Func1 FUnc2 
App1
App2
App3
\fi

\begin{tabular}{l|l|l|p{2.5cm}|p{2.5cm}|p{2.5cm}|p{2.5cm}|}
                & Rating    & Last update   & Supported devices & Media source & Connectivity & Pricing  \\
    \toprule
    SoundSeeder & 3.9       & 11/11 2016    & Android 4.1 and above, Java devices & Google Music, YouTube, online radio, local media & asdf & Free (limited) or 39.90 DKK \\
\end{tabular}

