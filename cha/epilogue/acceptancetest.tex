\chapter{Acceptance Test}\label{cha:acctest}
This chapter aims to test our application, in particular testing it in relation to the two apps which are similar to ours: AmpMe and SoundSeeder.
We have previously tested these, in \vref{sec:sota_test}, where we tested their offset, consistency and drift.
We test to see if we fulfilled the requirements we made in \vref{cha:requirement_elicitation} particularly: \ref{req:sync}, \ref{req:auto_off}, \ref{req:change_off}, \ref{req:pauseplay}, and \ref{req:drift}.
Also to see if we surpassed the milestone \ref{milestone:advanced_sync} we made in \vref{sec:milestones}.

In order to be fair we will test our app in the same way as we tested AmpMe and SoundSeeder, in order for them to be comparable.
Therefore the setup and procedure of this test will be the same as the one in \vref{sec:test_setup} and \vref{sec:test_procedure} respectively.

AmpMe and SoundSeeder both have manual sliders to adjust the synchronization on each device, however we do not have this as we rely on doing this automatically, therefor we cannot repeat the previous test for our app.
We will compare our automatic synchronization to the synchronization they do initially.

In total we will test the following questions from \vref{sec:sota_test} about our app:
\begin{itemize}
    \item Is the audio synchronized without any manual adjustment?
    \item Is the synchronization stable over time?
    \item Is the delay consistent between playbacks?
\end{itemize}

\section{Results}

\vref{fig:smus_pause_resume} shows the results of the next song test for our app.
Here we measure five data points, and take their average for each data point.
\subimport{figures/}{smus_pauseplay.tex}

\vref{fig:smus_drift_test} shows the results of the drift test for our app.
\subimport{figures/}{smus_drift.tex}

\section{Discussion}

\begin{description}
    \item[Is the audio synchronized without any manual adjustment?] \hfill \\
        This question should be considered in relation to to the requirement \vref{req:auto_off}, which states that the offset should be lower than 40 ms between the devices.
        During our tests, as shown in \vref{fig:smus_drift_test} offset is generally much lower than that.
        The figure \vref{fig:smus_drift_test_nofilter} contains all the datapoints, there there are 6 data points which are clear outliers.
        They outliers in the results are likely due to glitches in our time synchronization and TCP congestion.
        Preventing the glitches will be part of our future work in \vref{cha:future_work}.
        In order to better shown the general synchronization over time we have \vref{fig:smus_drift_test_filtered}, where we show the points in the [50;-30] ms range.
        If we do a linear regression on the filtered data (removing the 6 outliers), then the formula is: $f(x) = 0.0017x + 7.1149$, however the $R^2$ value is very low at $0.0013$.
        This is caused by the fact that we often resynchronize between the devices.
        Each time synchronization is completely independent, and we expect each of them to be close to correct, however we do not expect two time synchronizations in a row to be close.
        We imagine that it is possible to decrease the span of the offset by utilizing some or all of the techniques from \ac{NTP}, however we will explain this more in \vref{cha:future_work}.

        In total for our app the audio synchronization is close to $7$ ms on average, and non of the regular data is more than $40$ ms out of synchronization.
        This is well within the requirement we had which was $40$ ms in \vref{req:auto_off}.

    \item[Is the delay consistent between playbacks?] \hfill \\
        To test this we paused the playback, and resumed it, in our app this causes all devices to discard the buffer they previously had and rebuffer.
        However since the time synchronization in our app is independent of the playback, we did not expect that the offset would change any more than it would during playback.
        In \vref{fig:smus_drift_test} we show the results of 

    \item[Is the synchronization stable over time?] \hfill \\
        As discussed earlier our synchronization is close to perfect on average, but jumps around more than both AmpMe and SoundSeeder.
        Since we resynchronize often, we expect our synchronization to be stable, as was observed in the test.
\end{description}
