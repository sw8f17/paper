\chapter{Evaluation}\label{cha:conclusion}
In order to evaluate the success or failure of our project we first summarize our initial goals, milestones and problem statement.
We then discuss some of our choices and how it impacted our goals, and the situation of the application.
We also reconsider the milestones with the new data in regard to what we believe is a realistic end.
This is followed by a conclusion on the success of the application where we look at its capabilities in regard to our milestones, requirements and goals.
Finally we consider some limitations of our application, and what the effect of these limitations are on the end result.

\section{Summary}
%init problem and effects
Our initial problem was very broad and vague, encapsulating any system where synchronized sound streamed to multiple devices may be of use.
In order to specify the problem we explored two very different use cases, and two very different psychoacoustic effects.
The use case isolated playback felt underwhelming and compared to our other use case, increase volume and area, also seemed useful in a significantly lower number scenarios.
The increase volume and area also tied in nicely with one of the presented psychoacoustic effects, the precedence effect.
The other presented effect is 3D audio effect, this serves an entirely different purpose, and while the application may be able to support this effect the two effects cannot be used simultaneously. 
The two effects fits entirely different goals, while both interesting we chose to focus increasing the volume and area of multiple devices, using the precedence effect if possible.

%Sota
Upon deciding on a use case, we chose to explore products related to said use case.
We knew of some very successful multi-room audio speakers created by Sonos, which essentially we wanted to do, but with phones.
In the case of using phones we managed to find two decent applications attempting to do what we wanted to, however our tests revealed that these two applications were lackluster.
Using these applications as a benchmark, we wanted to produce an application which performed better.

%Problem Statement
With use case, related work and the precedence effect in mind we refined our problem to:
\begin{problemstatement}
    How can an Android application be used to achieve manipulated playback, wirelessly, across multiple smart phones, such that psychoacoustic effects can be utilized to enhance playback?
\end{problemstatement}

%Requirements
Derived from our problem statement and analysis of similar applications on the market we derived a number of requirements, both technical and \ac{UX} related.
These requirements concern features and thresholds made in regard to audible synchronization derived from the precedence effect.
We also established a number of non-requirements, these are requirements which may be important for specific scenarios, but that we choose to refrain from prioritising such that we can use our times on reaching our other goals.
In order to get a better sense of what needs to be done, and what we could possibly do with the information we gathered we made a set of milestones.
These milestones include both realistic and unrealistic goals segmented in a way such that we can simply continue developing till me meet a deadline, when we close on a deadline the milestones in correlation with our development method will lead us to whether or not the next milestone is feasible.


%%%% Disposition
%Initial thoughts?
%The Goal
%Use Case 
%Sota mentions perhaps?
%Non-reqs, why dis? why deez?
%Milestones, where did we expect to end up?

\section{Discussion}
%What are we discussing
While designing and developing the system we made certain choices for architecture, communication, clock synchronization etc.
In order to evaluate the final application we consider the impact of these choices, possible alternatives whether we knew of them when developing or learned of them later, and consider how a different choice may have affected the end application.

%External Architecture
In \cnameref{cha:architecture} we defined two different kinds of architectures, external and internal, related to communication and the internal application structure.
The internal architecture mostly affects our structure when developing and as such has a relatively low impact on the outcome of the application, therefore we focus on the external architecture.
During the design phase we concluded that the local centralized architecture would be the best idea, however for comparison purposes we ended up developing both a modified version of our remote centralized idea, as well as the local centralized idea.
The comparison relates closely to how we approach clock synchronization, so before going over the results let us have a look at how we approached synchronization.

\bigskip
%Clock Sync - highly coupled with the external architectures segment - TBH might wanna flip these as external hardware change is largely a result of us using SNTP and not NTP, if we were using and taking advantage of NTP, local SNTP would not be feasible
In \cnameref{cha:sync} we evaluated a couple of different techniques with \ac{NTP} being the most promising.
Any of the other solutions would have been largely experimental and more than likely resulted in an end result less accurate than what we have now.
However, during development we ended up switching from \ac{NTP} to \ac{SNTP}.
As the name suggests \ac{SNTP} is based upon \ac{NTP} but vastly simplified, \ac{SNTP} trades accuracy and drift alleviation for simplicity and resource requirements.
\ac{SNTP} was developed as an \ac{NTP} alternative for small computers and micro-controllers, which happens to fit the description of smartphones.
Furthermore we would not be taking full advantage of the utilities \ac{NTP} provides over \ac{SNTP} resulting i needless complexity.
In reality, \ac{SNTP} is actually a better fit than \ac{NTP} considering how we want to use it, in order to understand why, first let us consider some of the major advantages which makes \ac{NTP} more accurate than \ac{SNTP} and why they are of little benefit to our case.\cite{sntp_ntp}
\begin{description}
    \item [Clock Skew] \hfill \\
    Part of the algorithms \ac{NTP} incorporates are designed to alleviate clock skew.
    While effective in real-time systems, where the device clock needs to be synchronized, it has no impact on our application due to us not sending real-time commands, but rather timestamps for when an event should occur.
    \item [Drift Alleviation] \hfill \\
    \ac{NTP} also alleviates by drift, it does this by speeding up or slowing down the clock, effectively introducing drift in the opposing direction such that it evens out.
    As mentioned before, in order to adjust the clock through the application the device must be rooted, since we do not want to impose this requirement onto users, drift alleviation would be nullified as we cannot slow down or speed up the clock.
    \item [Seamless Adjustments] \hfill \\
    Similarly to how drift alleviation works, the same technique is used by \ac{NTP} for when the clock is only slightly out of sync, this way a step in time is not introduced, which is how \ac{SNTP} synchronizes the clock.
    However once again, this would require a rooted device and as such is irrelevant for our case.
    \item [Multiple Time References] \hfill \\
    Lastly \ac{NTP} queries multiple time servers for the clock, and through its algorithms determines which ones to use, and then averages these to find the time, whereas \ac{SNTP} simply queries one server, is the server unavailable it asks another.
    This is the part of \ac{NTP} which would have the largest impact on our specific use case.
\end{description}
\mnote{Perhaps we should move this per the no new info in conclusion convention}

So with some of the major advantages of \ac{NTP} not being of value to us, we chose to implement \ac{SNTP}.
This however also made us reconsider the fact that we were going for a modified centralized solution, since using \ac{NTP} would require internet.
With \ac{SNTP} not taking advantage of multiple time references, the major benefit from \ac{NTP} in our use case, we wanted to see the how the performance would compare, if we simply removed \ac{NTP} time servers from the equation.

As a result we altered our \ac{SNTP} implementation in order to accommodate the possibility of removing the internet requirement, by essentially using our local network master as the \ac{NTP} server.
By having both solutions available it further cemented our decision that a local centralized solution would be better than a remote centralized solution, even if the remote server would only be involved in time synchronization.\mnote{Use refs/numbers once the tests have been run}

\bigskip
%Implementation sepcifics / android.MedioCodec


%Milestone completion / no manipulation / psychoacoustic effect


%%%% Disposition
%Architecture choice
%Communications choice
%Sync choice
%Psychoacoustic choice
%Manipulation, didn't meet this milestone - somehow use milestones

\section{Conclusion}\mnote{test results are MIA, will need refactoring once I have them}
%%%% Disposition
%Manipulation, didnt make it
    %Is our offset even stable enough to support this?
%Test result eval, we beat SoTA bitches, or so i pressume
%Milestone considerations
\section{Limitations}
%%%% Disposition
%Random ass crashes
    %Stability
    %Manual offset
    %Psychoacoustic effects --> saving grace effect more like
