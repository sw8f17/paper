\subsection{Increased Volume and Area}
This category of use cases have the thing in common that it uses the synchronization to increase the volume of the audio, or the area in which it can be heard.
Some of the aspects of the use cases occur in several of them, for instance the requirement of precise synchronization.

Examples of use cases in this category could be the following:

\begin{description}
    \item[Social Gathering] \hfill\\
        At a social gathering or party, dependent on the event, the participants wants to be able to listen to loud music.
        If no dedicated sound system is available, phones can be used. 
        Phones have a low music volume and if some participants are talking and dancing, it can be hard to hear the music.

        Given the ability to play music synchronized across all their phones, the participants can use their phone and play the same music as one big speaker.
        This will make the perceived volume of the music louder to the listeners and the overall experience of listening to the music better.
        If some at the party or gathering just want to talk, they can either turn down the volume of their devices, mute the music, or turn the music off,
        making it quieter for them while the other participants still can hear the music.
        The participants should also be able to requests songs to the playlist.

        For all the participants to have the impression of only one speaker is playing, 
        the synchronization have to be precise enough so that there is no perceived difference in playback of the music between the devices.

    \item[Festival Scenario] \hfill\\
        At a music festival, there is often an area for the concerts and an area for people to stay and party before and after the concerts, often called a camp.
        These camps can stretch over a large area, dependent on the size of the festival.
        It is difficult to play music over the whole camp and it is possible that the festival guests wants to hear it at different volumes, to allow conversation etc.

        Many, if not all, participants have phones which can be used to play music.
        By having these phones connect to a device, which can send synchronized music to the phones,
        the festival participants can all play the same synchronized music,
        increasing the area covered by the music.
        As with the social gathering/party use case,
        the festival participants should be able to control the volume locally without affecting the rest.
        Furthermore everybody could requests songs to the playlist, so everybody can get an opinion on the music.

    \item[Multi-room Setup] \hfill\\
        When at home, people often listen to music.
        What if the music playback would be synchronous across all rooms of the home?
        Then there should not be a difference in synchronization between each room,
        it should be an experience of walking from room to room and listen to the exact same song.

        By having a system which can synchronize music, devices in each room can be connected to the sound system and play the music synchronized.
        This renders the area of the music larger and gives the experience of one speaker or sound system playing.
        The connected devices should be able to individually adjust the volume, mute or turn off the music,
        without having an effect at the other connected devices, if someone in a room do not want to listen to music. 
\end{description}
