%vim: fo-=t
\section{Android Apps}\label{sec:sota_apps}
In this section we investigate the state of the art for synchronized streaming of music between different wirelessly connected mobile devices.
We found three apps capable this:
\begin{itemize}
    \item SoundSeeder
    \item AmpMe
    \item Chorus
\end{itemize}

These apps are found by searching sources like Google\footnote{https://www.google.dk with search terms ``android sync music playback''}, YouTube\footnote{https://www.youtube.com/}, App Store\footnote{https://itunes.apple.com/dk}, and Google Play Store\footnote{https://play.google.com/store} and from recommendations on forums.
An additional criteria for an app to be considered here is that the app much not be abandoned by the developers.
Which we defined as being without an update since, at least, 2013, and being incompatible with new devices.

To clarify, Google Play Store and the App Store are the official places to install or buy apps for Android and iOS devices respectively.

The three apps all use a master/slave connection and use different terms for their setup.
For clarification we generalise these terms. 
We call the device which selects the music and streams it, the master, and the devices which connects to the master (SoundSeeder calls it ``Speaker''), are referred to as slaves.

\subsection{SoundSeeder}\label{subsec:soundseeder}
SoundSeeder\footnote{http://soundseeder.com} is an app made by JekApps.
The current version as of writing is 1.6.5.

It is compatible with Android 4.1 and above, for older versions of Android (2.2 -- 4.0),
another app called SoundSeeder Speaker makes it possible for a device to be used as a slave.

SoundSeeder also provides a Java application for compatibility with other platforms e.g. Windows, macOS and Linux, but does not support iOS and Windows Phone\cite{soundseeder_ios}.

The app consists of a top menu and a burger menu at the left side, as seen on \cref{fig:soundseeder_screenshot}.
In the burger menu, the user can choose the different playback possibilities and switch to slave mode.
The main view of the app is a music player, where the music can be controlled as any other music player.
To play music to other devices as a master, press the ``add music button'' in the top menu,
choose the source of the music and choose the preferred music, and press play.
\jjnote{Det virker måske lidt specifikt, men det kommer an på hvad vi bruger det til senere}

To connect to a playing master device, as a slave.
Select the speaker mode from the burger menu and if it finds the device, it connects automatically.
It can take a bit of time for the slave device to find the master device, which can seem confusing and make users think they need to connect manually with an IP address.

The process of joining, when the slave is given the time, is rather simple.
In regard to the user interface, it is clotted with features and menus, which makes the app hard to navigate.
The design of the app is old and not pretty.

SoundSeeder streams music via Wi-Fi,
which means that all phones have to be on the same network\cite{soundseether_faq}\jjnote{Der mangler måske noget her med NAT}, it is also possible to use an ad--hoc network (Android Hotspot).
In regard to the master's music source, it can be Google Music, online radio stations, UPnP and DLNA devices, local media, and YouTube if using semperVidLinks\footnote{\url{https://play.google.com/store/apps/details?id=com.semperpax.sempervidlinksFree}}, an app for extracting video links.
SoundSeeder also supports streams from external sources, e.g.\ a microphone or AUX device.
The supported media formats further depend on the used master device and its Android version.\cite{soundseether_faq}

SoundSeeder synchronizes the audio playback when a slave connects, but it can also be done manually on the slave device.
On \cref{fig:soundseeder_slider}, the manual synchronization adjustment for SoundSeeder can be seen.
This slider is used in the case that the music is not fully synchronized, and goes from $-400 ms$ up to $+400 ms$, in $10 ms$ increments.
Additionally there is an auto synchronization button in the top menu and on the slider menu window.

SoundSeeder is free to install but the free version only allows two slave devices to connect for up to 15 minutes at a time.
The app license costs 39.90 DKK.

\begin{figure}[h!]
    \centering
    \begin{subfigure}[b]{0.45\textwidth}
        \footnotesize
        \centering
        \frame{\includegraphics[width=0.65\textwidth]{img/sota/soundseeder.png}}
        \caption{When the app is opened.}\label{fig:soundseeder_screenshot}
    \end{subfigure}
    \hfill
    \begin{subfigure}[b]{0.45\textwidth}
        \footnotesize
        \centering
        \frame{\includegraphics[width=0.65\textwidth]{img/sota/soundseeder_slider.png}}
        \caption{The synchronization slider.}\label{fig:soundseeder_slider}
    \end{subfigure}
    \caption{Screenshots from SoundSeeder}\label{fig:soundseeder_screenshots}
\end{figure}

\subsection{AmpMe}\label{subsec:ampme}
The second app is AmpMe\footnote{\url{http://www.ampme.com}}, made by Amp Me Inc.
The current version of the app at the time of writing is 5.1.1.
It supports Android 4.1 or newer and iOS 9.0 or newer.

In AmpMe the group of devices is called ``a party''.
When AmpMe is opened, it either displays the nearby parties or encourage you to create your own.
If a party is found nearby, you can join it as a slave and play the master's music.
If no party is found, or you want to create your own, you can choose to start it by choosing between Spotify,
YouTube, your local music library, or SoundCloud as music source, as shown on \cref{fig:ampme_screenshot}.
When a source and music is chosen, a player appears with the music playing, and your device works as a master.

It is very intuitive to join a party, or host one yourself and play music.
The interface is modern, minimalistic, and pleasant to use and look at.

In order to stream music from a master to slaves, AmpMe requires an Internet connection.
This means that the connected devices can be on WiFi, mobile data or another network, as long there is Internet access.

In AmpMe, the music is automatically synchronized upon party creation, but if there is an offset, it can be synchronized manually each individual slave.
On \cref{fig:ampme_screenshot}, the slider to manually synchronize can be seen.
The slider goes from an offset of $-15$ to $+15$ arbitrary offset units, in $1$ increments.

AmpMe is free to use in both Google Play and the App Store.\cite{amp_faq}\cite{amp_play}\cite{amp_itunes}

\begin{figure}[h!]
    \centering
    \begin{subfigure}[b]{0.45\textwidth}
        \footnotesize
        \centering
        \frame{\includegraphics[width=0.65\textwidth]{img/sota/ampme.png}}
        \caption{When the app is opened.}\label{fig:ampme_screenshot}
    \end{subfigure}
    \hfill
    \begin{subfigure}[b]{0.45\textwidth}
        \footnotesize
        \centering
        \frame{\includegraphics[width=0.65\textwidth]{img/sota/ampme_slider.png}}
        \caption{The synchronization slider.}\label{fig:ampme_slider}
    \end{subfigure}
    \caption{Screenshots from AmpMe}\label{fig:ampme_screenshots}
\end{figure}

\subsection{Chorus}\label{subsec:chorus}
The final app is called Chorus\footnote{\url{https://play.google.com/store/apps/details?id=com.avrapps.chorus}} and is made by AVR APPS.
The current version is 2.1, at the time of writing.
It supports Android 4.0 or newer, AVR APPS mention an iOS app, but the App Store page returns that the app is not available in our region\footnote{\url{https://itunes.apple.com/app/chorus/id894014439}}.

To share music between devices, all devices have to be on the same network.
Which mean WiFi or a mobile hotspot is required.
Chorus only supports playback of local media on the master device.

When the app is opened, it looks like a regular music player, as seen on \cref{fig:chorus_screenshot}.
Music is added by pressing the small plus sign in the right bottom corner.
Which Chorus then streams to connected slave devices.
To connect to a master device playing music, you press the menu button in the right top corner, and press join.
Here a menu pops up with devices playing on the network, and if you select the device shown, it connects as a slave and starts playing.
We encountered a scenario where the app kept loading when attempting to join an active network, after a restart of the app it worked.

It is intuitive to use the app, it is very minimalistic hence easy to navigate.
The design is very minimalistic, and seems to follow the new ``Material Design'' guidelines set out by Google.

Chorus supports manual and automatic synchronization.
On \cref{fig:chorus_screenshot} the synchronization slider can be seen,
it slides from $-400ms$ to $+400ms$, in steps of $10 ms$.

The app is free to download and use without any ads.\cite{chrous_play}.

\begin{figure}[h!]
    \centering
    \begin{subfigure}[b]{0.45\textwidth}
        \footnotesize
        \centering
        \frame{\includegraphics[width=0.65\textwidth]{img/sota/chorus.png}}
        \caption{When the app is opened.}\label{fig:chorus_screenshot}
    \end{subfigure}
    \hfill
    \begin{subfigure}[b]{0.45\textwidth}
        \footnotesize
        \centering
        \frame{\includegraphics[width=0.65\textwidth]{img/sota/chorus_slider.png}}
        \caption{The synchronization slider.}\label{fig:chorus_slider}
    \end{subfigure}
    \caption{Screenshots from Chorus}\label{fig:chorus_screenshots}
\end{figure}

\subsection{App Comparison}\label{ssec:app_comparison}
To determine what makes these apps state of the art,
we look at their vital characteristics.
These vital characteristics are the user rating, last update date, supported devices, media sources, and connectivity options.
We look at these characteristics since we find them important to mobile apps and can use these characteristics to evaluate the apps.
There are also other characteristics of the apps, but we deem these are the most important ones.
In \cref{tab:sota_comp}, a comparison of the apps can be seen.

\jjnote{This is where I got to, here be dragons}
To start with Chorus, it does not support other devices than Android, narrowing down the supported devices compared to SoundSeeder and AmpMe.
Furthermore it only supports local media as music source, requiring all media to be played to be on the device.
Chorus also has a lower rating than SoundSeeder and AmpMe, which indicates dissatisfied users,
and the last update happened on 11/5 2015 indicating that it is abandoned.
On the other hand Chorus streams via WiFi and Android hotspot, just as SoundSeeder, and it has automatic and manual synchronization.
Since Chorus only supports Android, playback of local media and do not seem to further developed,
we do not deem it as state of the art.

In regard to SoundSeeder and AmpMe, they have their app ratings, Android support, YouTube and local media support in common.
AmpMe supports iOS devices, but SoundSeeder has a Java version which means that it supports all devices running Java, which is an advantage.
Therefore to be a part of the state of the art, it is important to support different types of devices.

In regard to supported devices, SoundSeeder supports more devices than AmpMe, but AmpMe supports Spotify.
Spotify has around 100 million users, which makes it an important source to support\cite{spotify_subscribers}.
Google Music have not released any official user numbers, but it is expected to be lower than Spotify\cite{googlem_subscribers}.
On the other hand SoundSeeder supports external sources, so a source supporting Spotify could be used with SoundSeeder that way.
This means that it is important to support popular and widely used music sources to be considered state of the art.

SoundSeeder supports streaming via WiFi or Android hotspot.
This means that all devices have to be on the same network and have connection to each other,
which can be an issue in larger network configurations, where clients can be restricted from communicating with each other,
which is the case at Aalborg University.
AmpMe does not have this restriction since it uses an Internet connection, this does have the restriction of using mobile data.
Furthermore if slow mobile data is used it can cause problems with the synchronization, but so can Internet via WiFi.
As both solutions have their respective restrictions we deem that both using WiFi/hotspot and Internet is a part of the state of the art.

Both SoundSeeder and AmpMe synchronizes automatically upon created connection.
In the case that the synchronization becomes skewed, both apps have the possibility of manually synchronize and to set a synchronization offset.
Since there is a need for manual synchronization options, this could mean that the automatic synchronization can at times be faulty,
but it is a good alternative to either disconnect and reconnect or wait for the automatic synchronization synchronize by itself.
Therefore to be a part of the state of the art, an app have to be able to both automatically and manually synchronize.
The actual performance of the synchronization, in SoundSeeder and AmpMe will be tested in a later section.


\begin{table}
    \centering
    \scalebox{0.75}{
    \begin{tabular}{l|l|l|p{2.2cm}|p{2.5cm}|p{2.6cm}|p{2.2cm}|}
                    & \multirow{2}{*}{Rating}    & \multirow{2}{*}{Last update}   & Supported devices & \multirow{2}{*}{Media source} & \multirow{2}{*}{Connectivity} & \multirow{2}{*}{Pricing}  \\
        \toprule

        \multirow{7}{*}{SoundSeeder} & \multirow{7}{*}{3.9 Play Store}    & \multirow{7}{*}{11/11 2016}    & \multirow{6}{*}{Android 4.1$\le$,} & Google Music, & \multirow{6}{*}{WiFi,} &  \multirow{6}{*}{Free (limited),} \\
    & & & \multirow{6}{*}{Java devices} & YouTube, & \multirow{6}{*}{Android hotspot} & \multirow{6}{*}{39.90 DKK} \\
    & & & & external device, & & \\
    & & & & UPnP, & & \\
    & & & & DLNA, & & \\
    & & & & online radio, & & \\
    & & & & local media & & \\

    \midrule

        \multirow{4}{*}{AmpMe} & \multirow{3}{*}{4.2 Play Store} & \multirow{3}{*}{30/01 2017} & \multirow{3}{*}{Android 4.1$\le$,} & Spotify, & \multirow{4}{*}{Internet} & \multirow{4}{*}{Free} \\
    & \multirow{3}{*}{4.0 App Store} & \multirow{3}{*}{01/02 2017} & \multirow{3}{*}{iOS 9.0 $\le$} & SoundCloud, & & \\
    & & & & YouTube, & & \\
    & & & & local media & & \\

    \midrule

        \multirow{2}{*}{Chorus} & \multirow{2}{*}{3.3 Play Store} & \multirow{2}{*}{11/5 2015} & \multirow{2}{*}{Android 4.0 $\le$} & \multirow{2}{*}{Local media} & \multirow{1}{*}{WiFi,} & \multirow{2}{*}{Free} \\
    & & & & & \multirow{1}{*}{Android hotspot} & \\

    \bottomrule

    \end{tabular}}
    \caption{Comparison between the apps as of the 13\textsuperscript{th} of February 2017.}\label{tab:sota_comp}
\end{table}

