%vim: fo-=t
\section{Android Apps}\label{sec:sota_apps}
In this section we investigate the state of the art for synchronized streaming of music between different wirelessly connected mobile devices.
We found three apps capable this:
\begin{itemize}
    \item SoundSeeder
    \item AmpMe
    \item Chorus
\end{itemize}

These apps are found by searching sources like Google\footnote{\url{https://www.google.dk} with search terms ``android sync music playback''}, YouTube\footnote{\url{https://www.youtube.com/}}, and Google Play Store\footnote{\url{https://play.google.com/store}} and from recommendations on forums.
An additional criteria for an app to be considered here is that the app must not be abandoned by the developers,
which we defined as being without an update since, at least, 2013, and being incompatible with new devices.

To clarify, Google Play Store is the official place to install or buy apps for Android.

The three apps all use a master/slave connection and use different terms for their setup.
For clarification we generalize these terms.
We call the device which selects the music and streams it, the master, and the devices which connects to the master (SoundSeeder calls it ``Speaker''), are referred to as slaves.

\subsection{SoundSeeder}\label{subsec:soundseeder}
SoundSeeder\footnote{\url{http://soundseeder.com}} is an app made by JekApps.
The  version as of \formatdate{9}{2}{2017} is 1.6.5.

It is compatible with Android 4.1 and above, for older versions of Android (2.2--4.0),
another app called SoundSeeder Speaker makes it possible for a device to be used as a slave.

SoundSeeder also provides a Java application for compatibility with other platforms e.g. Windows, macOS and Linux, but does not support iOS and Windows Phone\cite{soundseeder_ios}.

The app consists of a top menu and a burger menu at the left side, as seen on \cref{fig:soundseeder_screenshot}.
In the burger menu, the user can choose the different playback possibilities and switch to slave mode.
The main view of the app is a music player, where the music can be controlled as any other music player.
To play music to other devices as a master, press the ``add music button'' in the top menu,
choose the source of the music and choose the preferred music, and press play.

To connect to a playing master device, as a slave, select the speaker mode from the burger menu.
If it finds the device, it connects automatically.
It can take a bit of time for the slave device to find the master device, which can seem confusing and make users think they need to connect manually with an IP address.

The process of joining, when the slave is given the time, is rather simple.
The user interface is clotted with features and menus, which makes the app hard to navigate.
The design of the app is old and not pretty.

SoundSeeder streams music via Wi-Fi,
which means that all phones have to be on the same network\cite{soundseether_faq}, it is also possible to use an ad-hoc network (Android Hotspot).
The master's music source can be Google Play Music, online radio stations, UPnP and DLNA devices, local media, and YouTube if using semperVidLinks\footnote{\url{https://play.google.com/store/apps/details?id=com.semperpax.sempervidlinksFree}}, an app for extracting video links.
SoundSeeder also supports streams from external sources, e.g.\ a microphone or AUX device.
The supported media formats further depend on the used master device and its Android version.\cite{soundseether_faq}

SoundSeeder synchronizes the audio playback when a slave connects, but it can also be done manually on the slave device.
On \cref{fig:soundseeder_slider}, the manual synchronization adjustment for SoundSeeder can be seen.
This slider is used in the case that the music is not fully synchronized, and goes from $-400 ms$ up to $+400 ms$, in $10 ms$ increments.
Additionally there is an auto synchronization button in the top menu and on the slider menu window.

SoundSeeder is free to install but the free version only allows two slave devices to connect for up to 15 minutes at a time.
The app license costs 39.90 DKK\@.

\begin{figure}[h!]
    \centering
    \begin{subfigure}[b]{0.45\textwidth}
        \footnotesize
        \centering
        \frame{\includegraphics[width=0.65\textwidth]{img/sota/soundseeder.png}}
        \caption{When the app is opened.}\label{fig:soundseeder_screenshot}
    \end{subfigure}
    \hfill
    \begin{subfigure}[b]{0.45\textwidth}
        \footnotesize
        \centering
        \frame{\includegraphics[width=0.65\textwidth]{img/sota/soundseeder_slider.png}}
        \caption{The synchronization slider.}\label{fig:soundseeder_slider}
    \end{subfigure}
    \caption{Screenshots from SoundSeeder}\label{fig:soundseeder_screenshots}
\end{figure}

\subsection{AmpMe}\label{subsec:ampme}
The second app is AmpMe\footnote{\url{http://www.ampme.com}}, made by Amp Me Inc.
The version of the app as of \formatdate{10}{2}{2017} is 5.1.1.
It supports Android 4.1 or newer and iOS 9.0 or newer.

In AmpMe the group of devices is called ``a party''.
When AmpMe is opened, it either displays the nearby parties or encourage you to create your own.
If a party is found nearby, you can join it as a slave and play the master's music.
If no party is found, or you want to create your own, you can choose to start it by choosing between Spotify,
YouTube, your local music library, or SoundCloud as music source, as shown on \cref{fig:ampme_screenshot}.
When a source and music is chosen, a player appears with the music playing, and your device works as a master.

It is intuitive to join a party, or host one yourself and play music.
The interface is modern, minimalistic, and pleasant to use and look at.

In order to stream music from a master to slaves, AmpMe requires an Internet connection.
This means that the connected devices can be on WiFi, mobile data or another network, as long there is Internet access.

In AmpMe, the music is automatically synchronized upon party creation, but if there is an offset, it can be synchronized manually at each individual slave.
On \cref{fig:ampme_screenshot}, the slider to manually synchronize can be seen.
The slider goes from an offset of $-15$ to $+15$ arbitrary offset units, in $1$ increments.

AmpMe is free to use in both Google Play and the App Store.\cite{amp_faq}\cite{amp_play}\cite{amp_itunes}

\begin{figure}[h!]
    \centering
    \begin{subfigure}[b]{0.45\textwidth}
        \footnotesize
        \centering
        \frame{\includegraphics[width=0.65\textwidth]{img/sota/ampme.png}}
        \caption{When the app is opened.}\label{fig:ampme_screenshot}
    \end{subfigure}
    \hfill
    \begin{subfigure}[b]{0.45\textwidth}
        \footnotesize
        \centering
        \frame{\includegraphics[width=0.65\textwidth]{img/sota/ampme_slider.png}}
        \caption{The synchronization slider.}\label{fig:ampme_slider}
    \end{subfigure}
    \caption{Screenshots from AmpMe}\label{fig:ampme_screenshots}
\end{figure}

\subsection{Chorus}\label{subsec:chorus}
The final app is called Chorus\footnote{\url{https://play.google.com/store/apps/details?id=com.avrapps.chorus}} and is made by AVR APPS\@.
The version is 2.1, as of \formatdate{13}{2}{2017}.
It supports Android 4.0 or newer, AVR APPS mention an iOS app, but the App Store page returns that the app is not available in our region\footnote{\url{https://itunes.apple.com/app/chorus/id894014439}}.

To share music between devices, all devices have to be on the same network.
Which means WiFi or a mobile hotspot is required.
Chorus only supports playback of local media on the master device.

When the app is opened, it looks like a regular music player, as seen on \cref{fig:chorus_screenshot}.
Music is added by pressing the small plus sign in the right bottom corner.
Which Chorus then streams to connected slave devices.
To connect to a master device playing music, you press the menu button in the right top corner, and press join.
Here a menu pops up with devices playing on the network, and if you select the device shown, it connects as a slave and starts playing.
We encountered a scenario where the app kept loading when attempting to join an active network, after a restart of the app it worked.

It is intuitive to use the app, it is minimalistic hence easy to navigate.
The design is very minimalistic, and seems to follow the new ``Material Design''\footnote{\url{https://material.io/guidelines/}} guidelines set out by Google.

Chorus supports manual and automatic synchronization.
On \cref{fig:chorus_screenshot} the synchronization slider can be seen,
it slides from $-400ms$ to $+400ms$, in steps of $10 ms$.

The app is free to download and use without any ads.\cite{chrous_play}.

\begin{figure}[h!]
    \centering
    \begin{subfigure}[b]{0.45\textwidth}
        \footnotesize
        \centering
        \frame{\includegraphics[width=0.65\textwidth]{img/sota/chorus.png}}
        \caption{When the app is opened.}\label{fig:chorus_screenshot}
    \end{subfigure}
    \hfill
    \begin{subfigure}[b]{0.45\textwidth}
        \footnotesize
        \centering
        \frame{\includegraphics[width=0.65\textwidth]{img/sota/chorus_slider.png}}
        \caption{The synchronization slider.}\label{fig:chorus_slider}
    \end{subfigure}
    \caption{Screenshots from Chorus}\label{fig:chorus_screenshots}
\end{figure}

\subsection{App Comparison}\label{ssec:app_comparison}
To determine what makes these apps state of the art, we look at their vital characteristics.
These vital characteristics are the user rating, last update date, supported devices, media sources, and connectivity options.
In \cref{tab:sota_comp}, a comparison of the apps can be seen.

First let us start with Chorus, it supports the least devices, only supports local media, has the lowest rating with some explicitly bad reviews, and with the recent update being in 2015 it seems abandoned.
Only its connectivity options seems on par with the other two mentioned apps, however reviews mention that it is incredibly slow, as such we see no reason to continue with this app, nor can we deem it state of the art.

As for SoundSeeder and AmpMe, they have their app ratings, Android support, YouTube and local media support in common.
AmpMe goes slightly beyond SoundSeeder when it comes to media support, by also supporting Spotify and SoundCloud.
Spotify has around 100 million users, which makes it an important source to support\cite{spotify_subscribers}.
SoundSeeder supports Google Music, and while they have not released any official user numbers it is expected to be lower than Spotify\cite{googlem_subscribers}.

As for device support, AmpMe supports iOS devices, but SoundSeeder has a Java version which means that it supports all devices running Java.
Java support could prove advantageous in the case a computer was available as a media source, however this merely evens out the superior media support of Spotify in AmpMe.

As for connectivity, SoundSeeder supports streaming via WiFi or Android hotspot.
This may be an issue on larger and public networks where clients are often restricted from connecting with one another, although an Android hotspot can rectify the issue of a closed WiFi network.
AmpMe does not have this problem as it uses an Internet connection, however that does restrict one to internet and possible bandwidth issues.
As for the presented use cases in \cnameref{cha:establishing_use_cases}, either way of connecting the devices are acceptable.

Both SoundSeeder and AmpMe synchronizes automatically when the devices connect.
In the case that the synchronization becomes skewed, both apps have the possibility of manually synchronize and to set a synchronization offset.
The need for manual synchronization options could indicate that the automatic synchronization can at times be faulty, yet their creators find manual synchronization preferable to reconnecting.
The actual performance of the synchronization, in SoundSeeder and AmpMe will be tested in \cnameref{sec:sota_test}.

\renewcommand\tabularxcolumn[1]{m{#1}}
\renewcommand{\arraystretch}{1.5}
\begin{sidewaystable}
    \small
    \begin{tabularx}{\textwidth}{XXXXXXX}\toprule
        & Rating \newline (Out of 5) & Latest update & Supported \newline devices & Media source & Connectivity & Pricing \\\midrule
        SoundSeeder & 3.9 Play Store & 11/11 2016    & Android 4.1+ \newline Java Devices & Google Music \newline YouTube \newline external devices \newline UPnP \newline DLNA \newline Online radio \newline Local media & WiFi \newline Android Hotspot & Free (limited) \newline 39.90 DKK \\
        AmpMe & 4.2 Play Store \newline 4.0 App Store & 30/01 2017 & Android 4.1+ \newline iOS 9.0+ & Spotify \newline SoundCloud \newline YouTube \newline Local media & Internet & Free \\
        Chorus & 3.3 Play Store & 11/05 2015 & Android 4.0+ & Local media & WiFi \newline Android Hotspot & Free  \\\bottomrule
    \end{tabularx}
    \caption{Comparison between the apps as of the 13\textsuperscript{th} of February 2017.}\label{tab:sota_comp}
\end{sidewaystable}
