\subsection{Setup}
In order to perform these tests, we need to have a setup for it. 

Testing the offset between two audio sources is a signal processing
task. To make the signal processing easy and effective a clear signal is
needed. Additionally the signal needs to be synchronized between the
channels at recording time, since otherwise the analysis would be
mislead by desync in the recording. Luckily most computers include
a port for doing just that, a stereo microphone port.

The stereo microphone port in the test computer is a \ac{TRS} jack
connector, with a common ground between the two channels. To separate
the channels into separate mono tracks we used a \ac{TRS} to \ac{RCA}
splitter, and a custom \ac{RCA} to mono \ac{TS}, to capture the two
separate right\jjnote{Is it right} channels from the audio sources.

% Troels additions below
Furthermore we limit our tests to a Master/Slave/Slave setup, that is;
one phone acts as a master, and we perform the tests on two slaves
connected to it. This makes the two devices under test as simlar as
possible, since neither has the advantage of neither having the refence
sound. Ideally we would also record the refence, the Master, but since
we only have two input channels that isn't possible.

In the tests we use three Android phones:
\begin{description}
	\item[Master]{OnePlus One (Android 6.0.1, Kernel 3.4.112-cyanogenmod-g62d75e8)}
	\item[Slave 1]{Moto X Style (Android 6.0.1, Kernel 3.10.84-perf-gb67345b)}
	\item[Slave 2]{OnePlus Two (Android 6.0.1, Kernel 3.10.84-perf+)}
\end{description}

We will automate the mechnical part of the testing, such as doing the
analysis and controlling the variables. The scripts are written in
python, and uses the \ac{ADB} interface to control the android phones.
The test itself will be performed in an automatic fashion using a Python script
we wrote.  This ensure consistency and minimize the chances of error caused by
humans.
