\section{Procedure}\label{subsec:procedure}

\subsection{Test sample}
We use the song ``Goodness Gracious'' by ``Ellie Goulding''.  We didn't
want the apps to change song, possibly causing a resync, while testing.
To avoid that possibility we loop the song for 30 minutes within
a single mp3 file.

\subsection{Connecting}
Since the apps being tested are network connected systems, we have to
connect them before the test can begin. The connection procedure is
slightly different for each app. Common for both apps is that we use
3 devices to test, one is the ``master'' while the others are
``speakers''. The ``speakers'' are the ones connected to the computer,
and where we are measuring the offset.

\paragraph{AmpMe}
AmpMe allows the devices to connect over the public internet. We test
them on the AAU internal network, where they are technically connected,
but the internal network topology makes them unable to connect directly.
To connect speakers with the master, we start the master playing a song,
thereby starting a ``party'' and allowing the speaker devices to
connect. Once the speakers have connected we restart the song on the
master and wait for the app to resume playback.

\paragraph{Soundseeder}
Soundseeder does require the devices to be able to connect directly. We
fulfilled that requirement by having the master serve a wifi hotspot,
which allowed the devices to connect, while keeping the speakers
similarly configured. Soundseeder doesn't require a song to be playing
for it to connect. For the tests we connect the speakers to the master,
and only after they are connected begin playback.

\subsection{Volume}
The volume of the devices should not be particularly important, since
the method we use for correlation shouldn't care about the relative
amplitude. Regardless we will need to keep the volume low to avoid
clipping, since the microphone input is amplified. For the tests we keep
the volume levels of both devices at 3 steps above the minimum in
android.

\paragraph{Offset}
To test the offset aspect of the apps, we use the method described in
\cref{sec:test_setup}. We plug the devices into a computer, and take
5 sample, each of which are the median of 5 10-second measurements. The
measurements are correlated using the \ac{FFT} method also described in
\cref{sec:test_setup}.

\paragraph{Manual adjustment}
Testing the manual adjustments generally follow the same methodology as
the offset test. We plug the phones into the computer headphone jack,
and start the script. The script takes 5 measurements and moves the
slider one step for every iteration. It starts off moving the adjustment
slider to the right, until the maximum adjustment has been reached, it
then moves back to the initial position and repeats for the negative
adjustment values. After the last negative adjustment it returns to the
median slider position and records a final set of 5 measurements.

\paragraph{Playback consistency}
Playback consistency is, like the previous test, done by a script. The
script takes 5 measurements and then switches songs. It does this
5 times.  For AmpMe switching songs can be triggered by clicking the
previous song button, while Soundseeder requires the app to go to the
next song and then back to the previous.

\paragraph{Drift}
The drift of an app necessarily takes a while to test. Like the other
tests, a script is written to handle the testing part. The test is
accomplished by starting playback of a long song, created by repeating
a short song multiple times, and then measuring the offset over time. To
give us an idea of the drift over we take 5 measurements every 5 minutes
for 2 hours.
