\subsection{Procedure}\label{subsec:procedure}

Referring back to \jjwarning{Ref to the list from the whywetest.tex}, the
first aspect we wanted to test was offset between playing devices.

\paragraph{Offset}
To test the offset aspect of the apps, we use the method described in
\cref{subsec:test_setup}. We plug the devices into a computer, and take
5 measurements, each of which are the median of 5 measurements. The
measurements are correlated using the \ac{FFT} method also described in
\cref{subsec:test_setup}.

\paragraph{Manual adjustment}
Testing the manual adjustments generally follow the same methodology as
the offset test. We plug the phones into the computer headphone jack,
and start the script. The script takes 5 measurements and moves the
slider one step for every iteration. It starts off moving right, until
the maximum adjustment has been reached, it then moves back to the
initial position and repeats for the negative offset values. After the
last negative offset it return to the median position and records
another set of 5 samples.

\paragraph{Playback consistency}
Playback consistency is, like the previous test, done by a script. The
script takes 5 samples and then switches songs. It does this 5 times.
For AmpMe switching songs can be triggered by clicking the previous song
button, while Soundseeder requires the app to go to the next song and
then back to the previous.

\paragraph{Drift}
The drift of an app necessarily takes a while to test. Like the other
tests, a script is written to handle the testing part. The test is
accomplished by starting playback of a long song, created by repeating
a short song multiple times, and then measuring the offset over time. To
give us an idea of the drift over we take 5 samples every 5 minutes for
2 hours.
