\section{Procedure}\label{sec:test_procedure}
To make the tests reproducible we follow a specific and detailed
procedure for every test.

\begin{description}
\item[Offset] \hfill\\
To test the offset aspect of the apps, we use the method described in
\cnameref{sec:test_setup}. We plug the devices into a computer, and take
five samples, each of which are the median of five ten-second measurements. The
measurements are correlated using the \ac{FFT} method also described in
\cref{sec:test_setup}.

\item[Manual adjustment] \hfill\\
Testing the manual adjustments we generally follow the same methodology as
the offset test. We plug the phones into the computer headphone jack,
and start the appropriate script. The script takes five measurements and moves the
slider one step for every iteration. It starts off moving the adjustment
slider, seen on~\vref{fig:ampme_slider} and~\vref{fig:soundseeder_slider}, to the right, until the maximum adjustment has
been reached, it then moves back to the initial position and repeats for
the negative adjustment values. After the last negative adjustment it
returns to the median slider position and records a final set of
five measurements.

\item[Playback consistency] \hfill\\
Playback consistency is, similarly the previous test, done by a script. The
script takes five measurements and then switches song. It does this
five times. For AmpMe switching song can be triggered by clicking the
previous song button, while SoundSeeder requires the app to go to the
next song and then back to the previous.

\item[Drift] \hfill\\
The drift of an app necessarily takes a while to test. Like the other
tests, a script is written to handle the testing part. The test is
accomplished by starting playback of a long song, created by repeating
a short song multiple times, and then measuring the offset over time. To
give us an idea of the drift over we take five measurements every five minutes
for two hours.
\end{description}
