\subsection{Results}
In this section we present the results of the tests.
The interpretation of these is presented later.
For all results shown in tables; each of the measurements is the median of five. 
Furthermore it should be noted that the values were generally within $\pm 0.1 ms$ of each other. 

\subsubsection{SoundSeeder}
First we tested the SoundSeeeder app, using the setup and methods listed previously.\fxnote{Add cref when setup section exists}
In \vref{fig:soundseeder_test} the results of the first test, using the slider to manually adjust the offset, is shown. 
A total of 115 datapoints were collected in the test.
A trend--line for this data can be described by the function $f(x) = 0.9744x - 70.497$ with $R^2=0.9974$.
The trend--line is a linear reggression meaning the relationship between the sliders values and the measured values are nearly linear with 1 being exactly linear.
\subimport{}{soundseederfigure.tex}

In \vref{fig:soundseedersyncbutton} we test SoundSeeders synchronize button.
\subimport{}{soundseedersyncfigure.tex}

In \vref{fig:soundseedernextsongfigure} the result of the test, in which we change the playing song, are shown. 
\subimport{}{soundseedernextsongfigure.tex}

\subsubsection{AmpMe}
Secondly we tested the AmpMe app, using the exact same setup as the one for SoundSeeder. 
In \vref{fig:ampmedelay_test} the results of using the slider in AmpMe are presented.
A linear regression trend--line is drawn onto this data, the function for which is $f(x) = 23.211 * x + 131.260$ with $R^2 = 1.0$.
That means that the trend--line matches the data measured in a linear fashion. 
\subimport{}{ampmedelayfigure.tex}

In \vref{fig:ampmenextsongfigure} we show the result of the song change test. 
It is notable that the relative offset to previous synchronizations are close the the factor in the trend--line, however the cause is unknown. 
\subimport{}{ampmenextsongfigure.tex}

% drift tests, wait for new phones ...
