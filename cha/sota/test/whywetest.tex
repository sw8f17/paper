In \cnameref{cha:sota} we identified two apps, AmpMe and SoundSeeder, which are candidates for solving the initiating problem.
In this chapter we want to test whether they solve it to a satisfying degree.
That is to say we want to identify some parameters, and test to what degree the state of art achieves them.

We have identified the following three categories to test:
\begin{eletterate*}
    \item offset,
    \item consistency, and
    \item drift.
\end{eletterate*}

These three categories are related but slightly different.
Firstly, offset refers to whether or not there is an offset between different devices, e.g.\ is the playback of one device behind that of another.
Secondly, consistency refers to whether or not a potential offset is consistent as the playback is manipulated by something different from time, i.e.\ is the offset consistent when the song is changed, or the playback is paused and then resumed.
Thirdly, drift refers to whether the offset changes during playback over time.

Each of these are required for an app to fulfill the initial problem statement presented in~\cref{sec:initial_problem}.
Moreover, if the apps fail, in any of the categories, they are not fully suited as a solution for the problem at hand.
Concretely we want to answer the following questions in each category:
\begin{eletterate}
    \item Offset
    \begin{enumberate}
        \item Is the audio synchronized without any manual adjustment?
        \item Can the manual adjustment make them synchronized?
        \begin{enumberate}
            \item What does the slider do exactly?
            \item How much does each step affect the offset?
        \end{enumberate}
    \end{enumberate}
    \item Consistency
    \begin{enumberate}[resume]
        \item Is the delay consistent between playbacks?
    \end{enumberate}
    \item Drift
    \begin{enumberate}[resume]
        \item Is the synchronization stable over time?
    \end{enumberate}
\end{eletterate}

These questions are, to some degree, independent and an application can fulfill some or all of them.

