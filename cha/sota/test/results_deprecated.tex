\subsection{Results}
The data collected during the test are shown on \cref{fig:ampme_test} and \cref{fig:soundseeker_test}.

\paragraph{AMPme}
In the test of AMPme we used one Phone which had the media (server), and a single other phone (client) which acted as a client for it. 
Both phones were connected to the university WiFi network (AAU-1x).
We used their built in slider, as explained previously, to attempt synchronize the audio, it ranges from $-15$ to $+15$, we tested it with a stepping of $5$. 
The results of this is shown in \cref{fig:ampme_test}.
On the x-axis is the value of synchronization slides is shown, and the y-axis is the observed delay between the clients in milliseconds.
Each step in their slider is approximately $23.2191$ milliseconds.
Without any manual adjustment the delay between the two phone was $-351.208$, that is sound coming from the client was $351.208$ milliseconds after the audio from the server. 
After manually tuning using their slider we got to $-3.229$ millisecond difference at the $+15$ setting.  

\paragraph{SoundSeeker}
SoundSeeker requires all clients to be on the same WiFi network, and for them to have a direct connection. 
This is not possible on the university WiFi due to their setup, so we used a phone to create an ad--hoc network. 
The same phone which created the network also acted at the server in the SoundSeeker app. 
We then had two other phone connect to the ad--hoc network, and connect to the server in the app.
The app has a synchronize button, we pressed it on both clients and the resulting delay between the clients was $-90.896$, that is one of them was $90.896$ behind of the other. 
We then attempted to synchronize the audio on them by adjusting the slider in the app.
The slider in the SoundSeeker app has the unit ms (millisecond), and are in steps of $10$.  


\subimport{}{ampmefigure.tex}
\subimport{}{soundseekerfigure.tex}

\subsubsection*{Conclusion}

