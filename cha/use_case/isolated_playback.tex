\section{Isolated Playback}
Another category of use cases can be classified as \enquote{providing isolated yet synchronized playback between multiple devices}.
This category does not specify the limits for acceptable synchronization in and of itself, but it does require that the individual devices are synchronized without being able to listen to each other.

Two examples of use cases, which fit this category are:
\begin{description}
    \item[Silent clubbing/disco] \hfill\\
        While one might argue that an essential element of going to a club or disco is experiencing the loud music, try imagining a silent disco ---
        A club where the deafening beat is replace by nothing but the sound of people shuffling their feet across the dance floor and silence.
        It would be a significantly different experience than you would normally have at a club, and it would require guests to posses a compatible device and a pair of headphones.
        These compatible devices could for example be Android smart phones, and the club could provide this equipment along with headphones at the entrance.

        The guests at the club, would then be able to take part in \enquote{clubbing} by using their synchronized device and a pair of headphones.
        Because all guests experience an isolated music playback, the requirements for how precise the synchronization should be, are softer than in the previously described use cases.
        This means that the music any two guests hears does not have to be precisely synchronized, since they would never hear each others audio.
        However, the synchronization would still have to be precise enough for the guests dancing to seam somewhat in tune, i.e.~all devices should differ no more than $\pm 500 ms$ in synchronization.

        In this use case scenario, it would also be possible for the individual user to apply an equalizer to the music, and more importantly control their own volume.
    \item[Multilingual movie theater] \hfill\\
        % Translation on demand
        % Synchronization must match what is on the screen
        % Multiple audio streams
        % Possible to do large buffers
        % Device can listen to sound in movie theater and use for sync.
\end{description}

% In common: Data/audio signal comes from central device such as computer or dedicated hardware.
%            People must have additional hardware (headphones) which can isolate their own audio

