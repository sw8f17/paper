\section{Isolated Playback}
Another category of use cases can be classified as \enquote{providing isolated yet synchronized playback between multiple devices}.
This category does not specify the limits for acceptable synchronization in and of itself, but it does require that the individual devices are synchronized without being able to listen to each other.

Two examples of use cases, which fit this category are:
\begin{description}
    \item[Silent Clubbing] \hfill\\
        While one might argue that an essential element of going to a club or disco is experiencing the loud music, try imagining a silent disco ---
        A club where the deafening beat is replace by nothing but the sound of people shuffling their feet across the dance floor and silence.
        It would be a significantly different experience than you would normally have at a club, and it would require guests to posses a compatible device and a pair of headphones.
        These compatible devices could for example be Android smartphones, and the club could provide this equipment along with headphones at the entrance.

        The guests at the club, would then be able to take part in \enquote{clubbing} by using their synchronized device and a pair of headphones.
        Because all guests experience an isolated music playback, the requirements for how precise the synchronization should be, are softer than in the previously described use cases.
        This means that the music any two guests hears does not have to be precisely synchronized, since they would never hear each others audio.
        However, the synchronization would still have to be precise enough for the guests dancing to seam somewhat in tune, i.e.~guests should not feel that they are not listening to the same music as everyone else.

        In this use case scenario, it would also be possible for the individual user to apply an equalizer to the music, and more importantly control their own volume.
    \item[Multilingual Movie Theater] \hfill\\
        Most popular movies nowadays are translated into a myriad of different languages, mostly by way of subtitles but also by dubbing the dialogue in a different language.
        This rises a new problem, when watching a translated movie in the theater.
        Surely only one translation can be used, thereby making it more difficult to follow along if one is not fluent in the spoken or subtitled language.

        One solution could be what this use case example proposes.
        Making the audience in a movie theater able to choose between multiple different audio tracks, and listen to it through a headset.
        These different audio tracks could be different translations, but also uncensored audio or a narrating audio track for blind persons.

        In such a setup the synchronization would need to be precise, since the audio should match the picture.
        However, because screening a movie does not include \enquote{live} or unpredictable audio, the individual audio streams could be sent well ahead of playing,
        thereby relieving the need for fast and stable network connection.
        Moreover, to aid in the synchronization the individual devices could utilize the main audio being played from the movie theater's speakers to line up with.
\end{description}

These examples have two significant things in common.
Firstly, both of these require the involvement of the business or organization in charge of the club or movie theater.
This means that the data to be transmitted to the guests' or audience's devices, could be sent from something like a dedicated computer, i.e.~a device without limited memory and computational power.
Secondly, both examples require the users of the individual devices to have some kind of headphones, which is a requirement that should be considered, as it may restrict some people from using the system.
