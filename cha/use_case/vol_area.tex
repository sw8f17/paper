\section{Category: Increase Volume and Area}\label{sec:category_increase_volume_and_area}
A shared trait for use cases in this category is the use of synchronization to increase the volume of the audio, or the area in which it can be heard.
All use cases in this category rely on the synchronization being done well enough such that no difference in audio playback is perceived.

Examples of use cases in this category are:
\begin{description}
    \item[Social Gathering] \hfill\\
        At a social gathering or party, dependent on the event, the participants wants to be able to listen to loud music.
        If no dedicated sound system is available, phones can be used.
        Phones are generally only capable of playing music at low volume and if some participants are talking and dancing, it can be hard to hear what is being played.

        Given the ability to play music synchronized across all their phones, the participants can use their phones and play the same music as one device.
        This will make the perceived volume of the music louder to the listeners and the overall experience of listening to the music better.
        If some at the party or gathering just want to talk, they can either turn down the volume of their device, mute or turn the music off on their own phone;
        making it quieter for them while the other participants still can hear the music.
        The participants could also be able to requests songs to the playlist.

        For all the participants to have the impression of only one speaker is playing,
        the synchronization has to be precise enough so that there is no perceived shift in time in playback of the music between the devices.
        Furthermore, devices drifting out of synchronization should be automatically resynchronized, i.e.\ without the users interference.
        In conclusion, this use case must be able to utilize the aforementioned precedence effect.

    \item[Festival Scenario] \hfill\\
        At a music festival, there is often an area for the concerts and an area for people to stay before and after the concerts, often referred to as a camp.
        These camps can stretch over a large area, dependent on the size of the festival.
        It is difficult to play music over the whole camp and it is possible that the festival guests wants to hear it at different volumes, to allow conversation etc.

        Many, if not all, participants have phones which can be used to play music.
        By having these phones connect to a device, which can propagate music,
        the festival participants can play the music in a synchronized manner,
        effectively increasing the area covered by the music.
        As with the social gathering use case,
        the festival participants should be able to control the volume locally without affecting the rest of the participating devices.
        Furthermore, everybody should be able to request songs, such that everybody can influence the choice of music.

        In contras to the previously presented use case, this one does not need to exploit the precedence effect, since the area of the playback likely is too large for a listener to hear more than a couple of devices.
        Therefore, in this use case the focus would be on a seemingly seamless playback, which would fill the area with music.

    \item[Multi-room Setup] \hfill\\
        When at home, some people listen to music.
        What if the music playback was synchronous across all rooms of the home?
        Then there should not be a difference in synchronization between each room,
        it should be an experience of walking from room to room and listen to the same song.

        By having a system which can synchronize music, devices in each room can be connected to the sound system and play the music synchronized.
        This renders the area of the music larger and gives the experience of one speaker or sound system playing.
        The connected devices should be able to individually adjust the volume, mute or turn off the music,
        without affecting other connected devices.
\end{description}

This category is the most similar, of the use case categories, to the solutions covered in \cnameref{cha:sota}.
Especially \textbf{Multi-room Setup} represents what Sonos does to a satisfactory degree.
This being a more static scenario, as rooms rarely change, speakers make sense in this scenario making a more mobile implementation, e.g.\ smartphones, less needed.
As for the other two use cases, these cases are evidently what both AmpMe and SoundSeeker are designed for.
However, as revealed through our tests in \cnameref{sec:sota_test_conclusion}, these solutions are not satisfactory.

