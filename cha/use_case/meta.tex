\chapter{Establishing Use Cases}
\label{cha:establishing_use_cases}
In order to determine the requirements for the system we will develop, we identify a set of use cases, for which synchronized distributed playback of audio could be relevant.
We use brainstorming among the group members to initially recognize use cases, and then select the most interesting.
This brainstorming is influenced by the use cases which the apps in \cnameref{sec:sota_apps} conforms to.
The found use cases are then categorized, such that scenarios which may share some fundamental ideas or requirements are grouped together.

In each category we first describe the common traits, then give concrete examples of use cases.
The goal of establishing the use cases is not to choose a single use case or category to develop a system for, but utilise the potential requirements of each use case categories in the requirements elicitation.\mnote{change this so it fits with psychoacoustic section}
