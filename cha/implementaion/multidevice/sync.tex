\section{Implementing Synchronization}\label{sec:impl_sync}
In this section we discuss how we implement the methods of synchronization we discussed in \cref{cha:sync}.

In it we decided, in \cref{sec:sync_conclusion}, that NTP would suit us best.
However for practical reason we decided to implement \ac{SNTP}, which as the name suggests is a simplified version of \ac{NTP}, but does not require storing the state over time.
This solution requires each device to synchronize with an \ac{NTP}-server over the internet.

Since a persistent internet connection is not an ideal requirement for our application we also implemented \ac{SNTP} over the local network, we create using WiFi Direct, with the master working as a timeserver.
Doing this, would not synchronize each client to a precise global clock, but only the clock of the master device.
However for our application this it not a worry, our goal is to synchronize audio accurately across devices in our network.
This requires that the we have knowledge of the offsets between each client and the master, but it being accurate with any other device is irrelevant.

We decided to implement both of these, that is \ac{SNTP} over the internet and \ac{SNTP} over the WiFi Direct.
Then we will benchmark them to determine the solution which requires internet is better, and thusly justify the added requirement for the users.

As a side note, it is not possible for non-rooted Android devices to change the clock of the device, except with manual input in the setting menu.
For this reason we will not synchronize our clocks on the devices, but rather find the offset between the clock on a given device, and the clock of the \ac{NTP}-server.

\subsection{\ac{SNTP} over the Internet}


\subsection{\ac{SNTP} over WiFi Direct}

