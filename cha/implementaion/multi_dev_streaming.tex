\chapter{Multi-device Streaming}
In this chapter we will describe...

\section{Communication protocols}
In \cref{sec:external_architectures} it was discussed what type of external architecture should be used in the app.
The conclusion of that discussion was to go with a local centralized architecture.
This gives the requirement that the communication protocols, chosen for the communication between devices,
should adhear to the local centralized architecture.
This meant that the protocol used have to connect the devices, without the requirement of the Internet or external hardware, to a chosen master device.

In the attempt to fulfil the requirement posed in \cref{sec:external_architectures}, we found three protols: Nearby Messages API\footnote{\url{https://developers.google.com/nearby/messages/overview}}, Nearby connections API\footnote{\url{https://developers.google.com/nearby/connections/overview}}, and WiFi Direct\footnote{\url{http://www.wi-fi.org/discover-wi-fi/wi-fi-direct}}.
We will describe these protocols and choose on to implement in the app in the following text.

\subsection{Nearby Messages API}
The Nearby Messages API is a API by Google which lets one pass small binary payloads between internet connected Android and iOS devices. 
It uses a combination of Bluetooth, Bluetooth Low Energy, WiFi and near--ultrasonic audio to communicate between devices.
The Nearby Messages API uses Googles servers to relay the payloads between devices, hence requiring internet to be working. 
Furthermore since it had to connect to a Google server, we experienced delay between the messages,
which we estimate will make it impossible for us to fulfilling requirement \ref{req:manipulate} from \cref{cha:requirement_elicitation}.
Therefore we do not choose to implement the Nearby Messages API.\cite{nearby_messages}

\subsection{Nearby connections AP}


\subsection{WiFi Direct}


indledende med hvilket design vi vil implementere, referer til cha 8 architecture

historien om wifi direct, nearby connection api, nearby messaging api
hurtig disreard ad-hoc netværk - selv sætte wifi op, dele kode
nearbymistede hosten da den virker som router