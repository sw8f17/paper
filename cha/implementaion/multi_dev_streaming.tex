\chapter{Multi-device Streaming}
In this chapter we will describe...

\section{Communication protocols}
In \cref{sec:external_architectures} it was discussed what type of external architecture should be used in the app.
The conclusion of that discussion was to go with a local centralized architecture.
This gives the requirement that the communication protocols, chosen for the communication between devices,
should adhere to the local centralized architecture.
This meant that the protocol used have to connect the devices, without the requirement of the Internet or external hardware, to a chosen master device.

In the attempt to fulfil the requirement posed in \cref{sec:external_architectures}, we found three protols: Nearby Messages API\footnote{\url{https://developers.google.com/nearby/messages/overview}}, Nearby connections API\footnote{\url{https://developers.google.com/nearby/connections/overview}}, and WiFi Direct\footnote{\url{http://www.wi-fi.org/discover-wi-fi/wi-fi-direct}}.
We will describe these protocols and choose on to implement in the app in the following text.

\subsection{Nearby Messages API}
The Nearby Messages API is an API by Google which lets one pass small binary payloads between internet connected Android and iOS devices. 
It uses a combination of Bluetooth, Bluetooth Low Energy, WiFi and near--ultrasonic audio to communicate between devices.
The Nearby Messages API uses Google's servers to relay the payloads between devices, hence requiring internet to be working. 
Furthermore since it had to connect to a Google server, we experienced delay between the messages,
which we estimate will make it impossible for us to fulfilling requirement \ref{req:manipulate} from \cref{cha:requirement_elicitation}.
Therefore we do not choose to implement the Nearby Messages API.\cite{nearby_messages}

\subsection{Nearby Connections API}
The Nearby Connections API is an API by Google which lets one discover, connects and exchange messages with other devices on a network. 
It can exchange messages through WiFi in real--time.\cite{nearby_connection} 
The Nearby Connections API fulfil the requirements put by the external architecture in \cref{sec:external_architectures},
since internet is not required, and no external hardware is required as well.

When tried in a test environment, we used a WiFi ad--hoc network to test it,
since the network at Aalborg University is configured so devices can not connect to each other.
Normal WiFi can be used, but when there is no access to WiFi, for instance outside, an ad--hoc WiFi network must be used. 
We found an ad--hoc network to be an annoyance to set up with having to connect each device to the network.
When using the ad--hoc network hosted by a device, in this case a smartphone,
the device which hosted the network could not be a part of the network, but only route traffic.
This was a huge downside and it would be hard to justify that one device have to act as a router for the network.
Therefore we choose not to use the Nearby Connections API. 

\subsection{WiFi Direct}
WiFi Direct is a technology which enables devices with WiFi to directly connect to one another,
without having to use a traditional router, and it do not need internet to function.
WiFi Direct can be used to make one--to--one connection or to create a group of devices,
which can connect to each other simultaneously.\cite{wifi_direct}
WiFi Direct fulfil the requirements put by the external architecture in \cref{sec:external_architectures},
since internet is no requirement, and no external hardware is required. 

When tried in a test environment, WiFi Direct worked as expected,
and there was no requirement for one device to act as router, as with the Nearby Connections API.
Therefore WiFi Direct is chosen for the communication protocol in the app.

\subsection{Summary}
The requirements in \cref{sec:external_architectures} states that the communication protocol should work without internet and external hardware.
We found three different ways of connecting devices: Nearby Messages API, Nearby Connections API, and WiFi Direct.
The Nearby Messages API required internet and was slow, hence not fulfilling the requirements.
The Nearby Connections API required a device to act as router when an ad--hoc network was used, thus not being an ideal solution.
The last, WiFi Direct, did not require internet or external hardware, and when tested it worked as expected.
Therefore WiFi Direct is chosen to create the connection between the different devices. 
