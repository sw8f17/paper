%MediaService
    %Foundation

%MusicProvider
    %Generic Android stuff -> Lets us find the media Source


%MusicPlayerActivity
    %Main Activity -> Initializes, UI    

%Seperation
    %Playback -> Executed code
    %Model -> file access

%UI Baseactivity

%Music service and how it is controlled
    %Services, why a service, close app, connect controller, controller connects service and activity facilitating control structure;
    %Service is independant of app,
    %Gief controller to control thingie(activity/service)
    %Cross device will use controller.
    %You have App --> controlled by service, acessable without the app open --> from any activity/service
    %This is why we make a sync service, which can acess controller.
%Providers, simply adding providers i.e. spotify in the future

%MediacontrollerCompat, get from context, rest is androidMagicx

\section{Googlesamples Media Player(WIP)}
%%%Meta%%%
    %This section describes our implementation and use of googlesamples media player, as a foundation for our app.
    %This work fulfills our first milestone <quote.milestone.one>, and produces a foundation for which we will continue with our next milestone <quote.milestone.two>
    %the googlesamples media player which we are using, is a large codebase, in this section we will be covering the elements and design choices that are used.
    %These choices are relevant as they decide how we will approach further development atop this code base, we use a code base like this such that we can focus on what is required in order to utilize multiple devices, rather than spending time developing a local media player.
This section describes our implementation and use of googlesamples' media player\footnote{https://github.com/googlesamples/android-UniversalMusicPlayer} as a foundation for our app.
We use this sample such that we can focus our project on the device communication and psychoacoustic effects from the beginning, rather than building a music player.
The sample is a large codebase, as such we do not cover everything in the codebase, rather we present the elements and general design choices.
The googlesamples code foundations follows the android development guidelines, as such this section will introduce the basic android building blocks that we will use in further development of the application.
%%%MMeta%%%%

\subsubsection{Activity}
%MusicPlayerActivity
%BaseActivity
%%%GenPop%%%
An \texttt{activity} is the primary point of interaction between application and user.
Each different view is considered a different activity in the application, the media player we use have a fairly limited amount of activities, with \texttt{MusicPlayerActivity} being the main activity.
%%%GenPop%%%

The \texttt{MusicPlayerActivity} holds a \texttt{MediaController} and a \texttt{MediaBrowser}, these classes facilitate browsing media and basic playback control features.
A media player is rarely an application for which an activity is kept active.
An activity has three states, foreground, visible and background.
Having an application run, without its activity visible would be considered a background activity, however these are killable, and by extension so is the process.
For simplicity we will consider leaving an app having no activity in the visible or foreground state.
As such, we need some other way to keep the application alive, such that the application can be allowed to play music, without an activity being either in the state foreground or visible.
To do so we use a service.

%In order to accomplish this we need
%An application such as a mediaplayer is not something one necessarily wants in the foreground constantly.
%A process have four lifecycle states, foreground activity, visible activity, background activity and empty process.
%Not wanting our mediaplayer in the foreground this leaves us two lifecycles, background activity and empty process, however both of these are killable.
%To avoid our application dieing we create a service, which gives the process priority such that it will not be killed.

%onStop() --> Service should keep alive and interact through controller

\subsubsection{Service}
%MusicService
%MediaBrowserService
    %Lasting Services, i.e. important to keep running even with no active activty (started service - not bound to activity)
%%%GenPop%%%
A service is used to keep an app running in the background for any reason.
They serve to allow apps to perform long--running operations, which happens to be exactly what a media player does.
There are two primary distinctions between services, started services and bound services.
A started service tells the OS to keep it running, even when the user leaves the app.
A bound service is spawned by another app because it wants to utilize the service in some form.
Essentially a bound service is used in order for two processes to communicate, these however are only relevant as long as the application is relevant, as such if the application that caused the service to be spawn is left, then the service can be killed alongside the application.\cite{androidFundamentals}
%%%GenPop%%%

In googlesamples we have a started service, \texttt{MusicService}.
\texttt{MusicService} provides a \texttt{MediaBrowser} through a service.
It creates a \texttt{MediaSession}, in doing so it allows the client to create a \texttt{MediaController} which can send commands to the \texttt{MediaSession} and thereby perform playback controls without the app having to be actively used.


\subsubsection{Helpers}
\mnote{title??? and content??? i dno what to write4dis}
%Intent
%Mediacontroller
%%%GenPop%%%
With activities and services being seperate, yet part of the same application we need a way in which to propagate information between the two.
To do this we utilize helpers, support libraries, in the format of \texttt{Compat} classes.
%%%GenPop%%%

Specifically used for the basic media player are \texttt{MediaControllerCompat}, \texttt{MediaBrowserServiceCompat}, and \texttt{MediaSessionCompat}.
The aforementioned classes which share part of the names from these helpers, all extend these helper classes.
The helper classes are used to access features, such as communication, while also providing backwards--compatibility.

%Compat, facilitates communication, i.e. magically gives u controller
%Basics
    %We have an activity(app??)
    %We utilize services such that we can control the activity without it being "active"(app open)
        %We have a musicservice at present, a communication service will come later to support cross device commands
    %Controllers are used to give the services commands (so what exactly is a controller?)

%Code stuff
    %Mainactivity  ---> Initializing the activity (app launcher) --> UI -> inherits from BaseActivity
    %Playback   ---> The code executed, pause, seek, etc.
        %This is what is controlled by the service calls???
    %Model ---> Media source access(we made external to access audio on device) --> relates to MusicProvider somehow
        %This is providers??


%Uncovered
    %MediaBrowser?, this is related to activity, mediabrowser updates UI?