\section{Time-synchronization over WiFi Direct}
As we explained in the start of this section, we would consider it advantageous not to require our users to be connected to the Internet.
Implementing time-synchronization over WiFi Direct, independently of any third-party server on this internet would remove this requirement.

Our way of time-synchronization over WiFi Direct, uses the same methods as the method we just explained about \ac{SNTP} over the Internet, with the change that the master of the network would also act as an \ac{NTP}-server.

Implementing this requires us to add two protobufs, one for the request, and one for the response.
The request contains a single timestamp: the timestamp of the slave when it sent the request.
The response contains three timestamps: it sends the initial timestamp which it received from the slave, the timestamp for when it received it, and the timestamp for when it sends the response.
From this the slave can note the timestamp when it receives the response, and thus it has all four timestamps, and can calculate its clock offset, $\theta$, relative to the master clock.

Implementing this on the master is really simple, since it now is the master clock, the offset of it to the master clock is by definition zero.
Additionally it has to receive the requests and respond to them.
This is done by registering a callback which will be called whenever a request comes in, then record the two timestamps, build the response, and send it back.

The slave has to send the requests periodically, in order to avoid clock drift.
And like the master, it has to register a callback for when responses arrive, from which is can calculate the offset and save it.
Any part of the system which needs the offset can then ask for it, and will receive the latest recorded offset.

