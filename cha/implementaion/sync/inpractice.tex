\section{Using the Synchronization in the App}
Now that we have offsets, we can therefore convert local timestamps to be synchronized with the rest of the network.
For simplicity we decided to use the offset our synchronization gave us just before sending messages over the network, and just after receiving them.
That is, when we send a message, the timestamp in it is of the master clock, and when we receive a message, it is converted to the local clock.
This is illustrated in \ref{fig:sntp_example}.
Using the calculations for \ac{SNTP} offset calculation we get the following offsets:
$-1s925ms$ for the master (the blue clock) relative to the \ac{NTP}-servers time, and
$-0s937.5ms$ for the slave (the green clock) relative to the \ac{NTP}-server.

Say for example that we want to start playback in one second, $02s175ms$ (\ac{NTP} time), then the master would give that as the timestamp for the PlayCommand.
When the slave receives the message it will then translate the timestamp to its local time, by subtracting its offset:
$-0s937.5ms$ from the timestamp $02s175ms$: $02s175ms - (-0s937.5ms) = 3s112.5ms$.
That means that the slave should start the playback at $3s112.5ms$ on its own clock.
The master should start its playback at $02s175ms - (-1s925ms) = 4s100ms.$

\begin{figure}[bth]
    \centering
    \begin{tikzpicture}[auto, line cap=rect,line width=1pt]
        % Baggrund
        \filldraw [fill=GoogleLightBlue] (0,0) circle [radius=1cm];
        % Tick hver 3. time
        \foreach \angle in {0,90,180,270}
          \draw[line width=2pt] (0,0) -- ++(\angle:0.975cm);

        \foreach \angle in {0,90,180,270}
          \draw[line width=3pt, color=GoogleLightBlue] (0,0) -- ++(\angle:0.80cm);
        % Time viser
        %\draw[line width=1.8pt] (0,0) -- ++(150:0.5cm);
        % Minut viser
        \draw[line width=1.2pt] (0,0) -- ++(70:0.7cm);


        % Baggrund
        \filldraw [fill=GoogleAmber] (3,0) circle [radius=1cm];
        % Tick hver 3. time
        \foreach \angle in {0,90,180,270}
          \draw[line width=2pt] (3,0) -- ++(\angle:0.975cm);

        \foreach \angle in {0,90,180,270}
          \draw[line width=3pt, color=GoogleAmber] (3,0) -- ++(\angle:0.80cm);
        % Time viser
        %\draw[line width=1.8pt] (3,0) -- ++(150:0.5cm);
        % Minut viser
        \draw[line width=1.2pt] (3,0) -- ++(85:0.7cm);


        % Baggrund
        \filldraw [fill=GoogleLightGreen] (6,0) circle [radius=1cm];
        % Tick hver 3. time
        \foreach \angle in {0,90,180,270}
          \draw[line width=2pt] (6,0) -- ++(\angle:0.975cm);

        \foreach \angle in {0,90,180,270}
          \draw[line width=3pt, color=GoogleLightGreen] (6,0) -- ++(\angle:0.80cm);
        % Time viser
        %\draw[line width=1.8pt] (6,0) -- ++(150:0.5cm);
        % Minut viser
        \draw[line width=1.2pt] (6,0) -- ++(80:0.7cm);

        \node[draw, thick, fill=GoogleLightBlue]    at (0,-1.6) {Master};
        \node[]                                     at (0,-2.2) {$03s050ms$};
        \node[draw, thick, fill=GoogleAmber]        at (3,-1.6) {NTP-server};
        \node[]                                     at (3,-2.2) {$01s175ms$};
        \node[draw, thick, fill=GoogleLightGreen]   at (6,-1.6) {Slave};
        \node[]                                     at (6,-2.2) {$02s250ms$};
    \end{tikzpicture}
    \caption{Example of \ac{SNTP} over the Internet in action, only the second hand is shown.}
    \label{fig:sntp_example}
\end{figure}

This approach has some advantages and disadvantages, the biggest advantage is simplicity, it allows us to only think about time synchronization in one place, and not in the player itself.
However the disadvantage is that the longer we buffer data, the more outdated it gets.


% Hvor ofte skal vi synkronisere!?!

% eksempel på begge ?!?
