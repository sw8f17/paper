\section{Decoding MP3}

In \cnameref{cha:initial_development} we presented our initial
foundation for development, namely the Google sample implementation of
a universal media player app. In that implementation an Android
component, namely the \code{android.media.MediaPlayer}, is used to
control the playback of audio streams.

The \code{MediaPlayer} is a high level Android framework component,
which takes media files and plays them back using the standard android
audio path. It also provides control over the playback functions, such
as \code{seekTo}, \code{start} (commonly known as play), and
\code{pause}. The mediaplayer is squarely aimed at fulfilling the
requirements of most common usecases for most apps. Without providing
guarantees about timing and accuracy.

MP3 files aren't stored in a format directly relatable to time, and
is neither directly playable, nor can they be sliced arbitrarily to make
facilitate simple data transfer. In fact, even though MP3 files are
naturally decomposed into chunks, you need multiple chunks to start the
playback. These things together make it difficult to stream MP3 with the
precision and control we desire.

Instead we decide to transfer the sound data as raw \ac{PCM} encoded
byte arrays. \ac{PCM} data is directly related to audio produced, making
arbitrary slicing and complex mixing possible. Since the \ac{PCM}
encoded data is what the soundcard accepts. We'd have to decode into it
at some point, which makes transferring it directly over the network
a simple choice.

To reuse as much of the functionality already provided by the sample, we
replace the android \code{MediaPlayer} with one that we design and
control. Since the \code{MediaPlayer} is such a high level concept, we
are able to interpret all the commands in a way that makes sense for
synchronized playback.

Conceptually the android \code{MediaPlayer} fulfills 3 separate
responsibilities, with some ancillary responsibilities, that wont be
touched on here: Decoding, playback control, and presentation. In our
replacement we have structured them as can be seen on
\cref{fig:mediaplayerParts}. The app interacts with the \code{PlayBack
Control} which facilitates communication between the decoding and the
presentation.

\begin{figure}[ht]
	\centering
	\begin{tikzpicture}[
			every node/.style = {%
				draw,
				rectangle,
				rounded corners = 1mm,
				line width = 0.2mm,
				align = center,
				minimum width = 3cm,
				minimum height = 1.5cm,
				text = black,
			},
			hide/.style = {%
				draw = none,
				minimum width = 0cm,
				minimum height = 0cm,
			},
		]
		\node (pc) at (0,0) {Playback Control};
		\node[hide] (i) [above=1cm of pc] {};
		\node (d) [below left=1cm and 1cm of pc] {Decoding};
		\node (p) [below right=1cm and 1cm of pc] {Presentation};
		\draw[-] (i) -- (pc);
		\draw[-] (pc) -- (d);
		\draw[-] (pc) -- (p);
	\end{tikzpicture}
	\caption{Overview of the \code{MediaPlayer}}\label{fig:mediaplayerParts}
\end{figure}

\subsection{Decoder}
\label{subsec:decoder}

The decoder takes the audio in a supported format and turns it into
\ac{PCM} audio data. The \ac{PCM} data has some format, which the
decoder reads when it first opens a file.

Decoding an MP3 is a CPU bound task, mostly bottlenecked by CPU
performance over filesystem performance, and often having very few I/O
waits. For this reason Java, a high level language, doesn't fit the
problem well, due to its high CPU overhead. For this reason, and because
a free and complete library exists, we opted to go with the C library
libmpg123 to do the decoding.\ libmpg123 includes many optimizations in
hand coded assembly for many different architectures, with and without
floating point processors, and a general purpose C implementation, where
none of the hand optimized procedures apply. For this reason it's an
excellent choice for a fragmented ecosystem like android, where many
different CPU architectures and configurations have to be supported.

As said, the library is written for use in C. Fortunately Java has
support for calling native C and C++ libraries to Java objects, with
something called the \ac{JNI}. To connect libmpg123 we write a thin
layer which converts the data from Java methods and object parameters,
and passes it to the appropriate libmpg123 methods, with the expected
datatypes. Since libmpg123 has support for opening and processing
multiple files simultaneously, we create a separate object to hold the
file instance. This leaves us with two classes, the \code{MP3Decoder}
and the \code{MP3File}.\jjnote{I don't like how this turned out. Maybe
there's something better?} The \code{MP3Decoder} has the responsibility
of resolving android content URLs into file descriptors, opening them
with libmpg123, and filling out an object containing the media
information of the file. When a mp3file is opened a \code{MP3File}
object is returned, which has methods for reading from the MP3 file one
frame at a time. To keep the decoding snappy, we've opted to keep as
much of the data as possible in native C structs, which is associated
with the \code{MP3File} object though a pointer stored as a Java long.
Expanding the diagram shown in \cref{fig:mediaplayerParts}, with the
decoding just discussed creates a diagram as shown in
\cref{fig:mediaplayerPartsdecoding}.

\begin{figure}[ht]
	\centering
	\resizebox{\linewidth}{!}{%
		\begin{tikzpicture}[
				every node/.style = {%
					draw,
					rectangle,
					rounded corners = 1mm,
					line width = 0.2mm,
					align = center,
					minimum width = 3cm,
					minimum height = 1.5cm,
					text = black,
				},
				hide/.style = {%
					draw = none,
					minimum width = 0cm,
					minimum height = 0cm,
					rounded corners = 0mm,
				},
				textonly/.style = {%
					draw = none,
					minimum width = 0cm,
					minimum height = 0cm,
				},
				container/.style = {%
					inner sep=20pt,
					minimum width = 0cm,
					minimum height = 0cm,
					fit = #1,
				},
			]
			\node (pc) at (0,0) {Playback Control};
			\node[hide] (i) [above=1cm of pc] {};
			\node (p) [below right=1cm and 1cm of pc] {Presentation};
			\draw[-] (i) -- (pc);
			\draw[-] (pc) -- (p);


			\node (mf) [below left=1cm and 1cm of pc] {MP3File};
			\node (md) [left=0.5cm of mf] {MP3Decoder};

			\path (mf) -- coordinate[midway] (mm) (md);
			\node (ml) [below=1.5cm of mm] {libmpg123};

			\node (d) [container={(mf) (md) (ml)}] {};
			\node[textonly] () [above left] at (d.south east) {Decoder};

			\coordinate (dhit) at (mf |- d.north);

			\draw[-] (pc) -- (dhit);
			\draw[-] (dhit) -- (mf);
			\draw[-] (dhit) -- (md);
			\draw[-] (md) -- (ml);
			\draw[-] (mf) -- (ml);
		\end{tikzpicture}
	}
	\caption{Overview of the \code{MediaPlayer}, with detailed decoding}\label{fig:mediaplayerPartsdecoding}
\end{figure}

\subsection{Playback Control}\label{subsec:playbackcontrol}

The decoder is a passive component. It's only active while
processing a request from the active component owning it. In other
words, it doesn't run if no method is invoked. That means it requires
something else to drive the decoding. To better facilitate the
disconnected nature of playback required for the multiple playback
milestone, discussed in \jjnote{Ref til troels}, we want to establish an
ideal virtual player, which wont be tied up to the playback speed of the
underlying presentation component, but will instead schedule audio to
play based on the time to play a single MP3 frame worth of bytes, which
is given by the formula~\ref{eq:ntpb}.\ sampleSize in this equation is
the number of bytes in a single \ac{PCM} sample, in stereo with 16-bit
encoded \ac{PCM} data, this value is $4$.

\begin{equation}\label{eq:ntpb}
	\begin{split}
		nanosToPlayBytes(bytes, sampleSize, sampleRate) :=\\
		(bytes * 1,000,000,000) / (sampleSize * sampleRate)
	\end{split}
\end{equation}

By calculating the time to play the bytes, instead of relying on the
presentation component, we get an authoritative player, which we can
reliably control and tweak to fit the network, even if the master
devices presentation component has faults, such as playing too quickly
or too slowly.

This active proxy component, which we call the virtual player, has the
authority of the playback, meaning it is the place to ask where in the
audio stream the playhead is, how much is left, and it's the component
which decides what should be played next. Adding the virtual player to
the recurring figure creates \cref{fig:mediaplayerPartsvp}. Keep in mind
that the figure shows ownership, not communication. Even though the
virtual player only has an edge going to playback control, it's actually
a direct communicator between the decoder and the presentation.

The virtual player works by running a periodic task every calculated
frame time. At that time it calculates the current ``correct'' playhead
position, and schedules frame into the presentation component, until
preload condition if filled. One possible preload condition could be
that we must have scheduled 3 MP3 frames ahead of the current playback
position.

\begin{figure}[ht]
	\centering
	\resizebox{\linewidth}{!}{%
		\begin{tikzpicture}[
				every node/.style = {%
					draw,
					rectangle,
					rounded corners = 1mm,
					line width = 0.2mm,
					align = center,
					minimum width = 3cm,
					minimum height = 1.5cm,
					text = black,
				},
				hide/.style = {%
					draw = none,
					minimum width = 0cm,
					minimum height = 0cm,
					rounded corners = 0mm,
				},
				textonly/.style = {%
					draw = none,
					minimum width = 0cm,
					minimum height = 0cm,
				},
				container/.style = {%
					inner sep=20pt,
					minimum width = 0cm,
					minimum height = 0cm,
					fit = #1,
				},
			]
			\node (pc) at (0,0) {Playback Control};
			\node[hide] (i) [above=1cm of pc] {};
			\node (p) [below right=1cm and 1cm of pc] {Presentation};
			\draw[-] (i) -- (pc);
			\draw[-] (pc) -- (p);


			\node (mf) [below left=1cm and 1cm of pc] {MP3File};
			\node (md) [left=0.5cm of mf] {MP3Decoder};

			\path (mf) -- coordinate[midway] (mm) (md);
			\node (ml) [below=1.5cm of mm] {libmpg123};

			\node (d) [container={(mf) (md) (ml)}] {};
			\node[textonly] () [above left] at (d.south east) {Decoder};

			\coordinate (dhit) at (mf |- d.north);

			\draw[-] (pc) -- (dhit);
			\draw[-] (dhit) -- (mf);
			\draw[-] (dhit) -- (md);
			\draw[-] (md) -- (ml);
			\draw[-] (mf) -- (ml);

			\node (vp) [below=1cm of pc] {Virtual Player};

			\draw[-] (pc) -- (vp);
		\end{tikzpicture}
	}
	\caption{Overview of the \code{MediaPlayer}, with Virtual Player}\label{fig:mediaplayerPartsvp}
\end{figure}

\subsection{Presentation}\label{subsec:presentation}

The presentation component is where the audio is scheduled for playback
at a specific time, and then eventually presented to the listener at the
correct time.

\subsubsection{Audio Sink}\label{subsubsec:audiotrack}

The \code{AudioTrack}, an android framework component, accepts audio
written into a limited buffer and plays it at a steady, configurable
rate. Although for it's actively pushing data from its internal buffer
to the native soundcard buffer, the api that faces user apps is passive.
To add data the app has to call a write method, which will write the
buffer to the internal \code{AudioTrack} buffer. This means that the app
will have to keep track of when the audiotrack buffer is missing data,
and fill it back up, to avoid a buffer underflow, which manifests itself
as a pop in the audio for the listener. Since the underlying
\code{AudioTrack} and the virtual player might not be in sync, we can't
feed it with the data directly from that. We'd much rather have an
active interface that can query for new data, we call that interface
Audio Sink. The Audio Sink component is a thin wrapper around the
\code{AudioTrack} which converts the passive write api into an active
read based api which will ask for new data when necessary, specifying
with it the precise time that the new audio will be presented.

To help keep track of the playback position the \code{AudioTrack}
component includes functionality for periodic callback. We configure
that to fire every half buffer size, to effectively get double buffering
in the audio buffer. When around half of the buffer is gone, since it's
asynchronous we can't be sure, we calculate the available space in the
buffer. We want to queue more audio into that free space, but because we
are double buffering we also have to take into account the time it takes
to play the remaining buffer data when we determine what to queue up.
\code{AudioTrack} provides a convenient \code{getTimestamp} method which
gives you the \code{nanoTime} a specific frame was presented. From those
datapoints, the time we want to get audio for and how many samples we
want, we can determine which of the queued up samples from the virtual
player we need write right now.

\subsubsection{Audio Mixer}\label{subsubsec:audiomixer}

Now that we have two active components a passive component to
facilitate communication between the two is needed. In this case, such
a component is in charge of accepting the audio to schedule, for
a specific time, and giving it to the audio sink when needed. This
component is called the Audio Mixer.

Internally the audio mixer keeps an array of set size, which it mixes
the scheduled audio data into as needed, making sure it knows when it
has to be played. Since the \ac{PCM} coded audio has a fixed bitrate, it
can be encoded as a simple linear array, big enough to store the data
provided by the virtual player. The mixer knows what time the start of
the array represents, and can therefore calculate the point in time any
sample in the array corresponds to. It does the reverse operation in
order to determine the sample to start at for a given time. Adding the
audiomixer into the diagram we get what can be seen in
\cref{fig:mediaplayerPartspresenter}

\begin{figure}[ht]
	\centering
	\resizebox{\linewidth}{!}{%
		\begin{tikzpicture}[
				every node/.style = {%
					draw,
					rectangle,
					rounded corners = 1mm,
					line width = 0.2mm,
					align = center,
					minimum width = 3cm,
					minimum height = 1.5cm,
					text = black,
				},
				hide/.style = {%
					draw = none,
					minimum width = 0cm,
					minimum height = 0cm,
					rounded corners = 0mm,
				},
				textonly/.style = {%
					draw = none,
					minimum width = 0cm,
					minimum height = 0cm,
				},
				container/.style = {%
					inner sep=20pt,
					minimum width = 0cm,
					minimum height = 0cm,
					fit = #1,
				},
			]
			\node (pc) at (0,0) {Playback Control};
			\node[hide] (i) [above=1cm of pc] {};
			\draw[-] (i) -- (pc);


			\node (mf) [below left=1cm and 1cm of pc] {MP3File};
			\node (md) [left=0.5cm of mf] {MP3Decoder};

			\path (mf) -- coordinate[midway] (mm) (md);
			\node (ml) [below=1.5cm of mm] {libmpg123};

			\node (d) [container={(mf) (md) (ml)}] {};
			\node[textonly] () [above left] at (d.south east) {Decoder};

			\coordinate (dhit) at (mf |- d.north);

			\draw[-] (pc) -- (dhit);
			\draw[-] (dhit) -- (mf);
			\draw[-] (dhit) -- (md);
			\draw[-] (md) -- (ml);
			\draw[-] (mf) -- (ml);

			\node (vp) [below=1cm of pc] {Virtual Player};

			\draw[-] (pc) -- (vp);

			\node (pm) [below right=1cm and 1cm of pc] {AudioMixer};
			\node (ps) [right=0.5cm of pm] {AudioSink};

			\node (pl) at (ml -| ps) {AudioTrack};

			\node (p) [container={(pm) (ps) (pl)}] {};
			\node[textonly] () [above left] at (p.south east) {Presenter};

			\coordinate (phit) at (pm |- p.north);

			\draw[-] (pc) -- (phit);
			\draw[-] (phit) -- (pm);
			\draw[-] (phit) -- (ps);
			\draw[-] (ps) -- (pl);
		\end{tikzpicture}
	}
	\caption{Overview of the \code{MediaPlayer}, with detailed presenter.}\label{fig:mediaplayerPartspresenter}
\end{figure}
